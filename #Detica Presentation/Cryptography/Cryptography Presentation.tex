% !TEX TS-program = pdflatex
% !TEX encoding = UTF-8 Unicode

\documentclass{beamer}

\mode<presentation>
{
  \usetheme{Singapore}
  % or ...

  \setbeamercovered{transparent}
  % or whatever (possibly just delete it)
}

\usepackage[english]{babel}
\usepackage{calc}
\usepackage[utf8]{inputenc}

\usepackage{kpfonts}
\usepackage[T1]{fontenc}

\title[Cryptography] % (optional, use only with long paper titles)
{Cryptography}

\subtitle
{\& Quantum Computing} % (optional)

\author[Wainwright, Bloggs] % (optional, use only with lots of authors)
{J.~Wainwright}
% - Use the \inst{?} command only if the authors have different
%   affiliation.

\institute[Universities of Somewhere and Elsewhere] % (optional, but mostly needed)
{
  Department of Physics and Astronomy\\
  University of Birmingham}
% - Use the \inst command only if there are several affiliations.
% - Keep it simple, no one is interested in your street address.

\date[Short Occasion] % (optional)
{24th Feb 2012}

\subject{J. Wainwright - Inteview 24-02-12}
% This is only inserted into the PDF information catalog. Can be left
% out. 


% If you have a file called "university-logo-filename.xxx", where xxx
% is a graphic format that can be processed by latex or pdflatex,
% resp., then you can add a logo as follows:

% \pgfdeclareimage[height=0.5cm]{university-logo}{university-logo-filename}
% \logo{\pgfuseimage{university-logo}}

% Delete this, if you do not want the table of contents to pop up at
% the beginning of each subsection:
\AtBeginSubsection[]
{
  \begin{frame}<beamer>{Outline}
    \tableofcontents[currentsection,currentsubsection]
  \end{frame}
}


% If you wish to uncover everything in a step-wise fashion, uncomment
% the following command: 

\beamerdefaultoverlayspecification{<+->}


\begin{document}

\begin{frame}
  \titlepage
\end{frame}

\begin{frame}{Outline}
  \tableofcontents[pausesections]
  % You might wish to add the option [pausesections]
\end{frame}

\section{Introduction}

\begin{frame}{Definition}
	\begin{itemize}
		\item \emph{Cryptography} or \emph{cryptology}.
		\begin{itemize}
			\item Greek - \emph{kryptos} and \emph{logia} - the study of hidden writing.
			\item The practice and study of techniques for secure communication in the presence of third parties.
		\end{itemize}
		\begin{itemize}
			\item Involves a set of data to be sent which is;
			\begin{itemize}[<+->]
				\item Encrypted;
				\item Transferred;
				\item Decrypted;
				\item Read.
			\end{itemize}
		\end{itemize}
	\end{itemize}
\end{frame}

\begin{frame}{History}
	\begin{itemize}
		\item Pen and paper - effective if they can't read or write.
		\item Transposition ciphers
		\begin{itemize}
			\item ``hello world'' $\rightarrow$ ``ehlol owrdl''
		\end{itemize}
		\item Substitution ciphers
		\begin{itemize}
			\item ``hello world'' $\rightarrow$ ``ifmmp xpsme''
		\end{itemize}		
		
	\end{itemize}
\end{frame}

\section{Cryptography}

\subsection{Symmetric Key Cryptography}

\begin{frame}{Ciphertext \& Keys}
Some definitions
	\begin{itemize}
		\pause \item Normal information is called \emph{plaintext}.
		\item When the text is hidden its called \emph{ciphertext}.
		\item Messages are locked with keys:  
		\begin{itemize}
			\item Mathematical operation to hide text (e.g.\ transposition and substitution ciphers).
			\item The more complex the operation, the better hidden.
		\end{itemize}
	\end{itemize}
\end{frame}

\begin{frame}{Symmetric Key}
	\begin{centering}
		\pgfimage[width=0.8\paperwidth]{\string"Cryptography/symmetric key\string".pdf}
		\par
	\end{centering}
\end{frame}

\begin{frame}{Symmetric Key}
	\begin{itemize}
		\item Same key used to encrypt and decrypt.
		\item Benafits:
	 	\begin{itemize}
			\item Simple and easy to use.
			\item Quick encode and transfer.
		\end{itemize}
		\item Limitations:
		\begin{itemize}
			\item Only one level of security.
			\item Easy to break if intercepted.
		\end{itemize}
	\end{itemize}
\end{frame}

\subsection{Asymmetric Key Cryptography}

\begin{frame}{Asymmetric Key}
	\begin{itemize}
		\item Most common method used in the public domain.
		\item One key for encrypting, one for decrypting.
		\begin{itemize}
			\item Recipient makes locking and unlocking keys
			\item Locking key sent to sender
			\item Sender encrypts message using recieved key
			\item Ciphertext sent using public channels
			\item Recipient decrytps message using unlocking key
		\end{itemize}
	\end{itemize}
\end{frame}

\begin{frame}{Asymmetric Key}
	\begin{itemize}
		\item Most common method used in the public domain.
		\item One key for encrypting, one for decrypting.
		
		\begin{center}
			\begin{tabular}{c}
				\\
				\uncover<3->{Recipient makes locking and unlocking keys} \\
				\uncover<3->{$\downarrow$} \\
				\uncover<4->{Locking key sent to sender} \\
				\uncover<4->{$\downarrow$} \\
				\uncover<5->{Sender encrypts message using recieved key} \\
				\uncover<5->{$\downarrow$} \\
				\uncover<6->{Ciphertext sent using public channels} \\
				\uncover<6->{$\downarrow$} \\
				\uncover<7->{Recipient decrytps message using unlocking key}
			\end{tabular}
		\end{center}
	\end{itemize}
\end{frame}

\begin{frame}{Symmetric Key}
	\begin{centering}
		\pgfimage[width=0.8\paperwidth]{\string"Cryptography/asymmetric key\string".pdf}
		\par
	\end{centering}
\end{frame}

\section{Breaking In}

\begin{frame}{Brute Force}
	\begin{itemize}
		\item Nothing can be entirely resistant to attack.
		\item \emph{Attack} - interception of a message or key by any party other than the intended recipient.
		\item Brute Force tries every possible combination - last resort.
		\item Benafits:
	 	\begin{itemize}
			\item All messages can be broken.
			\item Simple algorithm to try everything sequentially.
		\end{itemize}
		\item Limitations:
		\begin{itemize}
			\item Time.
			\item Computing power needed to be reasonable.
		\end{itemize}
	\end{itemize}
\end{frame}

\begin{frame}{Unbreakable Codes}
	\begin{itemize}
		\item Nothing can be entirely resistant to attack.
		\item Larger keys = more time to break.
		\item ANother method availible - ``one-time pad''
		\item Benafits:
	 	\begin{itemize}
			\item All messages can be broken.
			\item Simple algorithm to try everything sequentially.
		\end{itemize}
		\item Limitations:
		\begin{itemize}
			\item Time.
			\item Computing power needed to be reasonable.
		\end{itemize}
	\end{itemize}
\end{frame}

\section*{Summary}

\begin{frame}{Summary}

  % Keep the summary *very short*.
  \begin{itemize}
  \item
    The \alert{first main message} of your talk in one or two lines.
  \item
    The \alert{second main message} of your talk in one or two lines.
  \item
    Perhaps a \alert{third message}, but not more than that.
  \end{itemize}
  
  % The following outlook is optional.
  \vskip0pt plus.5fill
  \begin{itemize}
  \item
    Outlook
    \begin{itemize}
    \item
      Something you haven't solved.
    \item
      Something else you haven't solved.
    \end{itemize}
  \end{itemize}
\end{frame}


\end{document}


