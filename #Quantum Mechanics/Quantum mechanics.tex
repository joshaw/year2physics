%% LyX 2.0.2 created this file.  For more info, see http://www.lyx.org/.
%% Do not edit unless you really know what you are doing.
\documentclass[english]{article}
\usepackage[T1]{fontenc}
\usepackage[latin9]{inputenc}
\usepackage{geometry}
\geometry{verbose,tmargin=3cm,bmargin=3cm,lmargin=3cm,rmargin=3cm,headheight=2cm,headsep=2cm,footskip=2cm}
\usepackage{fancyhdr}
\pagestyle{fancy}
\setlength{\parskip}{\medskipamount}
\setlength{\parindent}{0pt}
\usepackage{xcolor}
\usepackage{pdfcolmk}
\usepackage{babel}
\usepackage{array}
\usepackage{units}
\usepackage{pdfpages}
\usepackage{multirow}
\usepackage{amsmath}
\usepackage{amssymb}
\usepackage{fixltx2e}
\usepackage{graphicx}
\usepackage{esint}
\PassOptionsToPackage{normalem}{ulem}
\usepackage{ulem}
\usepackage[unicode=true,pdfusetitle,
 bookmarks=true,bookmarksnumbered=true,bookmarksopen=true,bookmarksopenlevel=4,
 breaklinks=true,pdfborder={0 0 0},backref=section,colorlinks=false]
 {hyperref}

\makeatletter

%%%%%%%%%%%%%%%%%%%%%%%%%%%%%% LyX specific LaTeX commands.
\newcommand{\noun}[1]{\textsc{#1}}
%% Because html converters don't know tabularnewline
\providecommand{\tabularnewline}{\\}
%% A simple dot to overcome graphicx limitations
\newcommand{\lyxdot}{.}

\providecolor{lyxadded}{rgb}{0,0,1}
\providecolor{lyxdeleted}{rgb}{1,0,0}
%% Change tracking with ulem
\newcommand{\lyxadded}[3]{{\texorpdfstring{\color{lyxadded}{}}{}#3}}
\newcommand{\lyxdeleted}[3]{{\texorpdfstring{\color{lyxdeleted}\sout{#3}}{}}}

\@ifundefined{date}{}{\date{}}
%%%%%%%%%%%%%%%%%%%%%%%%%%%%%% User specified LaTeX commands.
\renewcommand{\theequation}{\thesection.\arabic{equation}}
\numberwithin{equation}{section}
\usepackage{makeidx}
\makeindex
\usepackage{marginnote}
\let\marginpar\marginnote
\let\myMarginpar\marginpar
\renewcommand{\marginpar}[1]{\myMarginpar{%
\tiny#1}}

\makeatother

\begin{document}
\pagenumbering{roman}


\title{Quantum Mechanics}


\author{Josh Wainwright}

\maketitle
\tableofcontents{}

\pagebreak{}

\pagenumbering{arabic} 
\setcounter{page}{1}

Will assume knowledge of :
\begin{itemize}
\item Ideas from QM1
\item 2nd order differential equations
\item Complex numbers
\item Partial derivatives etc.
\end{itemize}

\part{QM1 Recap}
\begin{itemize}
\item Wave-particle duality, distinction depends on what is being measured,
e.g. photons 

\begin{itemize}
\item particle; photoelectric effect
\item wave; electron diffraction
\end{itemize}
\item Waves interfere or superpose - adding amplitudes rather than adding
intensities as would be expected with particles
\item Energy of quantum depends on its frequency;
\[
E=hf\;\Rightarrow\; E=h\nu
\]
Momentum of quantum depends on its wavelength, $\lambda$;
\[
p=\frac{h}{\lambda}
\]
where h=Plank's constant=$6.626\times10^{-34}\unit{Js}$
\end{itemize}

\section{Wave Picture of Matter}

\marginpar{E+R ch3}[-1cm]

Sine wave, 
\[
\psi(x,t)=A\sin(kx-\omega t)
\]
(wave along the positive x-axis with time period $\frac{2\pi}{\omega}$)

Alternate form: 
\begin{eqnarray*}
\psi(x,t) & = & Ae^{i(kx-\omega t)}\\
 & = & A(\cos(kx-\omega t)+i\sin(kx-\omega t)
\end{eqnarray*}
where $k=\frac{2\pi}{\lambda}$, wavenumber, and $\omega=2\pi f$,
angular frequency.
\[
E=\frac{h\omega}{2\pi}=\hbar\omega\qquad p=\frac{h}{2\pi\lambda}=\hbar k
\]



\subsection{Probability Interpretation}

Classical light is described by electromagnetic waves (Maxwell's equations)
\begin{description}
\item [{e.g.}] E field along the x-axis $\underline{E}(x)=\underline{E_{0}}\sin(kx-\omega t)$
\end{description}
Power of this wave is proportional to the $(amplitude)^{2}$
\[
P\propto E_{0}^{2}\;(\text{\ensuremath{{\textstyle Energy/sec}} perpendicular to the beam)}
\]
Quantum mechanics however sees the beam of light as consisting of
a finite number of particles, discrete photons, with energy$=\hbar\omega$,
\[
\Rightarrow Power\propto n\hbar\omega
\]
where n=number of particles detected per second.

To allow these two equations to match up and describe the same situation;
\[
n\propto(amplitude)^{2}\text{of wavefunction describing photons}
\]
This must also apply in quantum mechanics when n becomes very small,
i.e. when $n\ll1$, when the photons are detected separately. So this
now works as an average power over a length of time when many photons
are detected but quantum mechanics cannot predict the arrival of individual
photons.

This invariably leads to a probabilistic interpretation,
\[
(amplitude)^{2}\text{of the wave function}\propto\text{probability of a photon arriving per second}
\]



\subsection{Uncertainty in Quantum Mechanics}

A sine wave has a definite momentum, 
\[
p=\hbar k=\frac{h}{\lambda}
\]
But the amplitude is the same for all values of x, so there is an
equal probability of being anywhere in x, i.e. no information on the
position of the particle.

A localized particle can be described with a wave packet.

Now there is a finite probability of finding the particle in $\Delta x$
but zero elsewhere.

Here the wave packet has a wave of $\lambda$ or $k$. This can be
constructed by summing together a large number of sine and cosine
waves of different wavelengths and amplitudes (Fourier Analysis).
But now, by limiting the spacial range, some of the information regarding
the momentum has been lost. So as $k\rightarrow\bigtriangleup k$,
\[
p=\hbar k
\]
Rigorous analysis of this shows that the limit is Heisenberg's Uncertainty
principle,
\[
\bigtriangleup x\bigtriangleup p\geq\frac{\hbar}{2}
\]
The intrinsic quantum limit of to the information available arises
from the wave picture, not dependent on the measurement technique.


\part{The Schrödinger Equation}

\marginpar{E+R pg 125-136}[-1cm]

For a full description of an object with quantum mechanics, we need
an equation of motion. This is analogous to the classical wave equation
in 1 dimension,
\[
\frac{\partial^{2}\psi}{\partial x^{2}}=\frac{1}{v^{2}}\frac{\partial^{2}\psi}{\partial t^{2}}
\]
where $\psi$is the classical wave-function. If this were describing
a light wave, $v=c$ and $\psi$would be the equation of the electromagnetic
or magnetic field. Its a 2nd order differential equation and can be
derived from Newton's and Maxwell's equations.

The quantum mechanical version however cannot be derived from any
more basic postulates as it is one of the fundamental postulates in
quantum mechanics.


\section{Plausibility Argument}

To justify the form of Schrödinger's Equation (SE), it must satisfy
a number of constraints.
\begin{enumerate}
\item Energy and momentum equations,
\begin{eqnarray}
E & = & mc^{2}\nonumber \\
p & = & \hbar k
\end{eqnarray}
Energy conservation at non-relativistic speeds, 
\begin{eqnarray}
E & = & \frac{p^{2}}{2m}+V\\
\text{Total energy of system} & = & E_{k}+E_{p}\nonumber 
\end{eqnarray}
Differential equation must be linear to allow superposition if $\psi_{1}$and
$\psi_{2}$are each solutions of SE, then 
\[
\psi=a\psi_{1}+b\psi_{2}
\]
 (where a and b are constants) must be another solution. This leads
to the property of superposition. Combine these two,
\begin{equation}
\frac{\hbar^{2}k^{2}}{2m}+V=\hbar\omega
\end{equation}

\item For a free particle, $v=v_{0}=\text{constant},$ expect a $\sin+\cos$
wave solution with no preferred position as all positions are equivalent
to each other.
\end{enumerate}
So try free particle solution of the form,
\[
\psi(x,t)=A(\cos(kx-\omega t)+\gamma\sin(kx-\omega t))
\]
where $A,\,\gamma=\text{constants}$ to be determined.

Differentiate with respect to time,
\begin{eqnarray}
\frac{\partial\psi}{\partial t} & = & A\omega(\sin(kx-\omega t)-\gamma\cos(kx-\omega t))\\
\frac{\partial^{2}\psi}{\partial t^{2}} & = & -Ak(\cos(kx-\omega t)-\gamma\sin(kx-\omega t))
\end{eqnarray}
Comparing (4) and (5) with (3) suggests that the differential equation
to contain $k^{2}\,\omega$ terms,
\begin{equation}
\alpha\frac{\partial^{2}\psi}{\partial x^{2}}+V_{0}\psi=\beta\frac{\partial\psi}{\partial t}
\end{equation}
where $\alpha,\,\beta=\text{constants}$ fixed by equation (3). This
is very similar to the classical wave equation other than the classical
wave equation has a 2nd order differential with respect to time where
as the quantum wave equation has a 1st.

So sub (4), (5) into (6),
\begin{equation}
A\left(-\alpha k^{2}+V_{0}\right)(\cos(kx-\omega t)+\gamma\sin(kx-\omega t)=A\beta\omega(\sin(kx-\omega t)-\gamma\cos(kx-\omega t))
\end{equation}
This equation must be true for all values of x and t, 
\begin{eqnarray}
\text{For, }kx-\omega t & = & 0\nonumber \\
\Rightarrow\sin(kx-\omega t) & = & 0\nonumber \\
-\alpha k^{2}+V_{0} & = & -\beta\omega\gamma\\
\text{For, }kx-\omega t & = & \frac{\pi}{2}\nonumber \\
\Rightarrow\cos(kx-\omega t) & = & 0\nonumber \\
\Rightarrow(-\alpha k^{2}+V_{0})\gamma & = & \beta\omega
\end{eqnarray}
Divide (7)/(8)
\[
\Rightarrow\frac{1}{\gamma}=-\gamma
\]
\[
\gamma^{2}=-1,\:\gamma=\sqrt{-1}=\pm i
\]
The sign choice for $\gamma$ is completely arbitrary, by convention
$\gamma=+i$. So (4) becomes,
\begin{eqnarray*}
\psi(x,t) & = & A(\cos(kx-\omega t)+i\sin(kx-\omega t))\\
 & = & Ae^{i(kx-\omega t)}
\end{eqnarray*}
This is the equation for a free particle,\emph{
\[
\psi(x,t)=Ae^{i(kx-\omega t)}
\]
}From (8)
\[
-\alpha k^{2}+V_{0}=-i\beta\omega\qquad\because\gamma=i
\]
Match terms with (3)
\[
\frac{\hbar^{2}k^{2}}{2m}+V_{0}=\hbar\omega
\]
So for energy conservation $\alpha=-\frac{\hbar^{2}}{2m}$ and $\beta=i\hbar$.

Sub into (6)
\[
\frac{-\hbar^{2}}{2m}\frac{\partial^{2}\psi}{\partial x^{2}}+V\psi=i\hbar\frac{\partial\psi}{\partial t}
\]
This is the time dependent form of the Schrödinger Equation
\[
\frac{-\hbar^{2}}{2m}\frac{\partial^{2}\psi}{\partial x^{2}}+V\psi=i\hbar\frac{\partial\psi}{\partial t}
\]
This equation holds for all functions of the potential, V, and not
just for the free particle with potential $V_{0}$ as was argued in
the construction of the equation. This is a postulate that is agreed
with by experimental data.

SE is a complex second order differential equation and the solution,
$\psi(x,t)$ is generally also a complex function.


\section{Interpretation in Terms of Real Measurements}

\marginpar{E+R pg 134-1501}[-1cm]

The wave-function $\psi$ is a complete description of the quantum
object. It contains all the information that can be know about it
subject to Heisenberg's Uncertainty Principle. $\psi$ is complex
and so cannot be measured experimentally. But we can apply mathematical
operators to extract physical properties.


\subsection{Probability Density Function (PDF)}

Define $P(x,t)dx=$probability of measuring the particle in the range
$x\rightarrow x+dx$. $P(x,t)$ must obey certain conditions. It must
be
\begin{itemize}
\item real
\item non-negative
\item normalized so that $P\geq1$ does not occur.
\end{itemize}
In quantum mechanics,
\begin{eqnarray*}
P(x,t) & = & |\psi|^{2}=\psi*\psi\\
\psi & = & a+ib\qquad\text{where }a,b\in\mathbb{R}\\
|\psi|^{2} & = & (a+ib)(a+ib)=a^{2}+b^{2}>0,\in\mathbb{R}\\
\therefore|\psi|^{2} & > & 0,\in\mathbb{R}
\end{eqnarray*}
Normalization depends on the physical situation being described. The
P over the whole of space must be =1 so that there is a 100\% probability
of measuring it somewhere.
\[
\Rightarrow\int_{-\infty}^{\infty}|\psi|^{2}dx=1
\]
\uline{\noun{Ex}}

\[
\psi(x,t)=\begin{cases}
A\cos(x)e^{-i\frac{Et}{\hbar}} & -\pi<x<\pi\\
0 & \text{otherwise}
\end{cases}
\]
{[}If $\psi_{1}$is a solution of SE then so is $\psi=A\psi_{1}$(the
original wave-function multiplied by a constant){]}
\begin{eqnarray*}
\int_{-\pi}^{\pi}\left(A\cos(x)e^{-i\frac{Et}{\hbar}}\right)\left(A\cos(x)e^{i\frac{Et}{\hbar}}\right)dx & = & 1\\
|A|^{2}\int_{-\pi}^{\pi}\cos^{2}(x) & = & 1\\
|A|^{2}\frac{1}{2}\int_{-\pi}^{\pi}(1+\cos(2x))dx & = & 1\\
|A|^{2}\pi & = & 1\\
A & = & \frac{1}{\sqrt{\pi}}\text{\:(Chosen A to be real)}\\
\Rightarrow\psi(x,t) & = & \begin{cases}
\frac{1}{\sqrt{\pi}}\cos(x)e^{\frac{-Et}{\hbar}} & -\pi<x<\pi\\
0 & \text{otherwise}
\end{cases}
\end{eqnarray*}
For a beam of particles, the normalization constant depends on the
beam density.


\subsection{Expectation Value}

Quantum Mechanics cannot predict the outcome of a single measurement,
it can only give a probability. We can calculate an average value
over many repeated measurements. This is the expectation value. These
repeated measurements must be of the exact same quantity so that the
wave-function is identical.
\[
<x>=\int_{-\infty}^{\infty}xP(x,t)dx
\]
This is simply the probability, P, weighted against the value of the
displacement, x. In quantum mechanics,
\[
<x>=\int_{-\infty}^{\infty}\psi*\psi dx
\]
$<x>$ is the average, not a single measurement.

\uline{\noun{Ex}}

\begin{eqnarray*}
\psi & = & A\cos(x)e^{\frac{-iEt}{\hbar}}\\
\Rightarrow<x> & = & \int_{-\pi}^{\pi}\left(A\cos(x)e^{\frac{-iEt}{\hbar}}\right)x\left(A\cos(x)e^{\frac{iEt}{\hbar}}\right)dx\\
 & = & A^{2}\int_{-\pi}^{\pi}x\cos^{2}(x)dx=0
\end{eqnarray*}
 Since $x\cos^{2}(x)$ is an odd function and the integration is symmetric
about $x=0$. This form is similar for other functions of x,
\begin{eqnarray*}
<x> & = & \int_{-\infty}^{\infty}\psi*x^{2}\psi dx\\
<f(x)> & = & \int_{-\infty}^{\infty}\psi*f(x)\psi dx
\end{eqnarray*}
However there are many quantities that might be needed, that cannot
be written as functions of x, e.g the momentum, which would not be
allowed by HUP if it dependent only on x. We can express these are
mathematical operators,
\begin{eqnarray*}
\hat{\mathcal{O}}f(x) & = & g(x)\\
\text{eg }\frac{d}{dx} & \left(x^{2}\right)= & 2x
\end{eqnarray*}
In quantum mechanics, an operator extracts physical information from
the wave-function.


\subsubsection{Momentum Operator:}

\[
\hat{P}=-i\hbar\frac{\partial}{\partial x}
\]


\uline{\noun{Ex}}

Momentum of a free particle, $\psi=Ae^{i(kx-\omega t)}$,
\begin{eqnarray*}
\hat{P}\psi & = & -i\hbar\frac{\partial}{\partial x}\left(Ae^{i(kx-\omega t)}\right)\\
 & = & -i\hbar ikAe^{i(kx-\omega t)}\\
 & = & \hbar kAe^{i(kx-\omega t)}\\
\Rightarrow\hat{P}\psi & = & \text{constant}*\psi\qquad\text{constant}=\hbar k
\end{eqnarray*}
Here the constant is equal to the momentum, $p=\hbar k$.


\subsubsection{Total Energy Operator:}

\[
\hat{E}=i\hbar\frac{\partial}{\partial t}
\]


\uline{\noun{Ex}}

Total energy of a free particle,
\begin{eqnarray*}
\hat{E}\psi & = & i\hbar\frac{\partial}{\partial t}\left(Ae^{i(kx-\omega t)}\right)\\
 & = & \hbar\omega Ae^{i(kx-\omega t)}\\
 & = & \hbar\omega\psi\\
\Rightarrow\hat{E}\psi & = & \text{constant}*\psi,\qquad\text{constant}=\hbar\omega
\end{eqnarray*}



\subsubsection{Kinetic Energy Operator:}

\begin{eqnarray*}
\hat{E_{k}}=\hat{T} & = & \frac{\hat{P}^{2}}{2m}\\
 & = & \frac{1}{2m}\left(-i\hbar\frac{\partial}{\partial x}\right)\left(-i\hbar\frac{\partial}{\partial x}\right)\\
 & = & -\frac{\hbar^{2}}{2m}\frac{\partial^{2}}{\partial x^{2}}
\end{eqnarray*}


\uline{\noun{Ex}}

Kinetic energy of a free particle,
\begin{eqnarray*}
\frac{\hat{P}^{2}}{2m}\psi & = & -\frac{\hbar^{2}}{2m}\frac{\partial^{2}\psi}{\partial x^{2}}\\
 & = & -\frac{\hbar^{2}}{2m}(-k^{2})\psi=\frac{\hbar^{2}k^{2}}{2m}\psi\\
\Rightarrow\hat{E_{k}}\psi & = & \text{constant}*\psi\qquad\text{constant}=\frac{\hbar^{2}k^{2}}{2m}
\end{eqnarray*}
From these operators we can rewrite the Schrödinger Equation in a
more compact form,
\begin{eqnarray*}
-\frac{\hbar^{2}}{2m}\frac{\partial^{2}\psi}{\partial x^{2}}+V\psi & = & i\hbar\frac{\partial\psi}{\partial t}\\
\frac{\hat{P}^{2}}{2m}\psi+V\psi & = & \hat{E}\psi\\
\mathcal{\hat{H}\psi} & = & \hat{E}\psi
\end{eqnarray*}
where $\mathcal{\hat{H}}$ is the Hamiltonian operator.

A quantum mechanical operator extracts physical information form a
wave-function.

e.g. for $\psi=Ae^{i(kx-\omega t)}$, $\hat{P}\psi=\hbar k\psi=p\psi$
\begin{eqnarray*}
\text{operator}\times(\text{function}) & = & \text{constant}\times(\text{function})\\
\hat{\mathcal{O}}\psi_{i} & = & a_{i}\psi_{i}
\end{eqnarray*}
Here, $\psi_{i}$ is the eigenfunction of the operator $\hat{\mathcal{O}}$
corresponds to the observable $\mathcal{O}$, it represents a quantum
eigenstate with a definite value for the observable $\mathcal{O}$.
$a_{i}$ is the eigenvalue of $\hat{\mathcal{O}}$, gives value of
$\mathcal{O}$ measured for the state $\psi_{i}$. The complete set
of eigenvalues $a_{i}(i=1,2,3\ldots,n)$ gives all the possible values
of the observable $\mathcal{O}$ that can be measured for this system.

So for example, $\psi=Ae^{i(kx-\omega t)}$ is an eigenfunction of
the momentum operator $\hat{P}$ and of the energy operator $\hat{E}$.
These then produce the eigenvalues of $p=\hbar k$ and $E=\hbar\omega$.


\subsection{Expectation Value for Operators}

For any observable, $\mathcal{O}$, operator $\hat{\mathcal{O}}$
expectation value is given by
\begin{eqnarray*}
<\mathcal{O>} & = & \int_{-\infty}^{\infty}\psi*\hat{\mathcal{O}}\psi dx
\end{eqnarray*}
So for the case where x is the observable, the operator is just x
and the operation carried out would be $x\psi$ and $\hat{x}\equiv x$.
Here however, when dealing with the operator expectation value, the
order of the integral matters as,
\[
\psi*\hat{\mathcal{O}}\psi\neq\psi\hat{\mathcal{O}}\psi*\neq\hat{\mathcal{O}}\psi\psi*
\]
The operator must be carried out on the original wave-function.

\uline{\noun{Ex}}

For the wave-function give by $\psi=\begin{cases}
A\cos(x)e^{-i\frac{Et}{\hbar}} & -\pi\leq x\leq\pi\\
0 & \text{otherwise}
\end{cases}$
\begin{eqnarray*}
<\hat{E_{k}}> & = & <\frac{p^{2}}{2m}>=\int_{-\pi}^{\pi}\hat{E_{k}}A\cos(x)e^{-i\frac{Et}{\hbar}}\psi*\\
 & = & \int_{-\pi}^{\pi}\left(A\cos(x)\right)\left(\frac{-\hbar^{2}}{2m}\frac{\partial^{2}}{\partial x^{2}}A\cos(x)\right)dx\\
 & = & \frac{\hbar^{2}}{2m}\int_{-\pi}^{\pi}A^{2}\cos^{2}(x)dx\\
<\hat{E_{k}}> & = & \frac{\hbar^{2}}{2m}
\end{eqnarray*}



\part{The Time Independent Schrödinger Equation}

\marginpar{E+R pg 150-155}[-1cm]

Often $V(x,t)=V(x)$ and so there is a static potential energy function.
In this case, the full Schrödinger Equation can be simplified,
\[
i\frac{\hbar}{2m}\frac{\partial^{2}}{\partial x^{2}}\Psi(x,t)+V(x)\Psi(x,t)=i\hbar\frac{\partial}{\partial t}\Psi(x,t)
\]
Try the solution $\Psi(x,t)=\Psi(x)\phi(t)$
\[
\Rightarrow\left(-\frac{\hbar^{2}}{2m}\frac{\partial^{2}}{\partial x^{2}}\psi+V\psi\right)\phi=\psi\left(i\hbar\frac{d\phi}{dt}\right)
\]
Divide through by $\Phi=\psi\phi$
\[
\Rightarrow\left(\frac{-\hbar^{2}}{2m}\frac{d^{2}\psi}{dx^{2}}+V\psi\right)\frac{1}{\psi}=\frac{1}{\phi}\left(-i\hbar\frac{d\phi}{dt}\right)=G=\text{constant}
\]
{[}LHS = only operators and functions of x, RHS = only operators and
functions of t{]}

If $X(x)=T(t)$, then only true for all $x$ and $t$ if $X(x)=T(t)=\text{constant}$.
So this 2nd order differential equation is now two ordinary differential
equations which can be solved. 

Time Dependant:
\begin{eqnarray*}
\frac{1}{\phi}i\hbar\frac{d\phi}{dt} & = & G\\
i\hbar\frac{d\phi}{dt} & = & G\phi
\end{eqnarray*}
But 
\begin{eqnarray*}
\hat{E} & = & i\hbar\frac{d\phi}{dt}=E\phi\\
\phi & = & Ae^{-\frac{Et}{\hbar}}\text{ [This is the time dependant part of }\Phi(x,t)\text{]}
\end{eqnarray*}
Space Dependant:
\begin{eqnarray*}
\frac{1}{\psi}\left(-\frac{\hbar^{2}}{2m}\frac{d^{2}\psi}{dx^{2}}+V\psi\right) & = & G=E\\
-\frac{\hbar^{2}}{2m}\frac{d^{2}\psi}{dx^{2}}+V\psi & = & E\psi\text{ [This is the time indepentant form of }\Psi(x,t)\text{]}
\end{eqnarray*}
Solution $\psi(x)$ (spacial part of the wave-function) must be compared
with $\phi(t)$ to get the full wave-function.
\[
\Phi(x,t)=\psi(x)e^{-i\frac{Et}{\hbar}}
\]
Time independent Schrödinger Equation is an ordinary differential
equation and is \noun{real}. Solution $\psi(x)$ may be real or complex.


\section{Solutions of TISE - Potential Step}

\marginpar{E+R pg 181-198}[-1cm]

Form of $\psi(x)$ depends on the form of $V(x)$. The simplest case
is when $V=V_{0}=\text{constant}$ (free particle)
\begin{eqnarray*}
-\frac{\hbar^{2}}{2m}\frac{d^{2}\psi}{dx^{2}}+V_{0}\psi & = & E\psi\\
\frac{d^{2}\psi}{dx^{2}} & = & -2m\left(\frac{E-V_{0}}{\hbar^{2}}\right)\psi\text{ [Simple harmonic oscillator equation]}
\end{eqnarray*}
This is a well know ODE with solutions of the form,
\begin{eqnarray*}
\psi & = & Ae^{ikx}+Be^{-ikx},\quad k=\frac{\sqrt{2m(E-V_{0})}}{\hbar}\\
\text{Equivalently }\psi & = & A\cos(kx)+B\sin(kx)
\end{eqnarray*}
Solve TISE for various $V(x)$ which are not uniform,
\[
\text{force}=\frac{dV}{dx}\neq0
\]
The simplest case is the potential step.


\subsection{Case when $V_{0}<E$}

\noindent \begin{center}
\begin{minipage}[t]{1\columnwidth}%
\noindent \begin{center}
\includegraphics{\string"Quantum Mechanics/Potential Step\string".pdf}
\par\end{center}%
\end{minipage}
\par\end{center}

This constitutes two regions of potential joined by a step at $x=0$.
This is not a physically realistic situation but a simplification.
Here $0<V_{0}<E$, where $E$ is the energy of the particle coming
from the left. Classically, the particle would be transmitted as $E>V_{0}$
so would continue towards $x\rightarrow\infty$; only being slowed
down. However,
\[
-\frac{\hbar^{2}}{2m}\frac{d^{2}\psi}{dx^{2}}=(E-V)\psi
\]



\subsubsection{Solve TISE for $V_{0}<E$}

We already know the solution for $V=0$ and for $V=V_{0}$ separately
as these are just TISE with a constant potential energy function.
\[
x\leq0,\; V=0,\;\psi_{1}(x)=Ae^{ik_{1}x}+Be^{-ik_{1}x}\qquad k_{1}=\frac{\sqrt{2mE}}{\hbar}
\]
\[
x>0,\; V=V_{0},\;\psi_{2}(x)=Ce^{ik_{2}x}+De^{-ik_{2}x}\qquad k_{2}=\frac{\sqrt{2m(E-V_{0})}}{\hbar}
\]
By definition, $V_{0}<E$, so $k_{2}<k_{1}$. So $\hbar k_{2}<\hbar k_{1}$,
which means the momentum after $x=0$ is less than before so the particle
slows down when it hits the potential barrier.

Need physical boundary conditions to find A, B, C and D;
\begin{enumerate}
\item $De^{-ik_{2}x}$ would represent a traveling reflected wave from the
right hand side, though there is nothing to be reflected from, so
can argue that this wave would not exist, so $D=0,$
\[
\Rightarrow\psi_{2}=Ce^{ik_{2}x}
\]

\item Continuity considerations at boundary, $x=0$,

\begin{enumerate}
\item Wave-function must be continuous else $p\propto\frac{d\psi}{dx}\rightarrow\infty$
\item Wave-function must have a continuous gradient $\frac{d\psi}{dx}$,
else kinetic energy $\propto\frac{d^{2}\psi}{dx^{2}}\rightarrow\infty$
\end{enumerate}

At $x=0$
\begin{enumerate}
\item $\psi_{1}(0)=\psi_{2}(0)$


$A+B=C$

\item $\frac{d}{dx}\psi_{1}(0)=\frac{d}{dx}\psi_{2}(0)$


$ik_{1}(A+B)=ik_{2}C$


$A-B=\frac{k_{2}}{k_{1}}C$

\end{enumerate}

$\Rightarrow2A=\left(1+\frac{k_{2}}{k_{1}}\right)C,\qquad C=\frac{2k_{1}}{k_{1}+k_{2}}A$


$\Rightarrow B=C-A=\left(\frac{2k_{1}}{k_{1}+k_{2}}-1\right)A,\qquad B=\left(\frac{k_{1}-k_{2}}{k_{1}+k_{2}}\right)A$

\end{enumerate}
In general, A, B and C must all have non zero values so have some
transmission and some reflection of the wave at the step, unlike the
classical particle which would be 100\% transmitted it $E>V_{0}$.
But this is similar to a classical \noun{wave }where reflection and
refraction both occur at the boundary. So QM particle exhibits wave
like properties.


\subsection{Transmission and Reflection Coefficients}

T = probability of transmission (beyond $x>0$)

R = probability of reflection

Flux = number density in the beam $\times$ velocity. Number density
$\propto|\psi|^{2}=\psi*\psi$, velocity $\propto k$.
\begin{eqnarray*}
\text{\ensuremath{\Rightarrow}Flux} & \propto & k|\psi|^{2}
\end{eqnarray*}
Transmitted probability,
\begin{equation}
T=\frac{\text{Transmitted flux}}{\text{Incident flux}}
\end{equation}
\[
T=\frac{k_{2}|C|^{2}}{k_{1}|A|^{2}}=\frac{k_{2}}{k_{1}}\left(\frac{2k_{1}}{k_{1}+k_{2}}\right)^{2}=\frac{4k_{1}k_{2}}{\left(k_{1}+k_{2}\right)^{2}}\quad\text{(For \textsc{this} particle)}
\]
Reflected probability, 
\begin{equation}
R=\frac{\text{Reflected flux}}{\text{Incident flux}}
\end{equation}
\[
R=\frac{k_{1}|B|^{2}}{k_{1}|A|^{2}}=\frac{|B|^{2}}{|A|^{2}}=\left(\frac{k_{1}-k_{2}}{k_{1}+k_{2}}\right)^{2}\quad\text{(For \textsc{this} particle)}
\]
The net flux is conserved, so $\sum(T+R)=1$

\begin{minipage}[t]{1\columnwidth}%
\noindent \begin{center}
\includegraphics{\string"Quantum Mechanics/Transmission with V0\string".pdf}
\par\end{center}%
\end{minipage}

If $V_{0}=0$, then $k_{2}=k_{1}$, then $T=1$ and $R=0$ (this is
the case when there is no step)

If $V_{0}=E$, then $k_{2}=0$, then $T=0$ and $R=1$ (full reflection)


\subsection{Case when $V_{0}>E$}

Classically, the particle would be reflected 100\% of the time with
perfect certainty. The region $x>0$ would be energetically forbidden
$\text{(}E_{p}\text{ barrier}>E_{k}\text{ of incident particle)}$


\subsubsection{Solve TISE for $V_{0}>E$}

\[
x\leq0,\; V=0,\;\psi_{1}(x)=Ae^{ik_{1}x}+Be^{-ik_{1}x}\qquad k_{1}=\frac{\sqrt{2mE}}{\hbar}
\]
\[
x>0,\; V=V_{0},\;\psi_{2}(x)=Ce^{-ik_{2}x}+De^{-ik_{2}x}\qquad k_{2}=\frac{\sqrt{2m(E-V_{0})}}{\hbar}
\]
This is identical to the situation where $V_{0}<E$. Now however,
when $V_{0}>E$, 
\[
k_{2}=\frac{\sqrt{2m\left(E-V_{0}\right)}}{\hbar}=a+ib
\]
. So the wavenumber is imaginary. So rewrite this as,
\begin{eqnarray*}
k_{2} & = & \frac{-\sqrt{2m\left(V_{0}-E\right)}}{\hbar}\\
k_{2} & = & iK,\qquad\text{where }K=\frac{\sqrt{2m\left(V_{0}-E\right)}}{\hbar}\in\mathbb{R}>0
\end{eqnarray*}
\[
\Rightarrow\psi_{2}(x)=Ce^{-Kx}+De^{Kx}
\]
Generally, the wave function of a particle is any region that is energetically
forbidden has this real exponential form (not oscillatory). Use boundary
conditions to determine the integration constants A, B, C and D.

As $x\rightarrow\infty$, $e^{Kx}\rightarrow\infty$. But $\psi_{2}(x)$
must remain finite because $|\psi|^{2}\propto\text{probability}$.
So put $D=0$.
\[
\Rightarrow\psi_{2}(x)=Ce^{-Kx}
\]

\begin{enumerate}
\item This time $D\neq0$
\item Continuity considerations at the step $x=0$ are the same. $\psi$
is continuous, 


At $x=0$
\begin{enumerate}
\item $\psi_{1}(0)=\psi_{2}(0)$


$A+B=C$

\item $\frac{d}{dx}\psi_{1}(0)=\frac{d}{dx}\psi_{2}(0)$


$ik_{1}(A-B)=-KC$

\end{enumerate}

\begin{eqnarray*}
\Rightarrow2A & = & \left(1+\frac{iK}{k_{1}}\right)C\\
\Rightarrow2B & = & \left(1-\frac{iK}{k_{1}}\right)C
\end{eqnarray*}
Reflection coefficient 
\[
R=\left|\frac{B}{A}\right|^{2}=\left|\frac{1-\frac{iK}{k_{1}}}{1+\frac{iK}{k_{1}}}\right|^{2}=1,\quad\Rightarrow T=0
\]


\end{enumerate}
So for the quantum particle, we also get 100\% reflection, the same
as for a classical particle. C, however, is non zero, so $\psi_{2}\neq0$
in the region $x>0$. So there is a finite probability of measuring
the particle in the region $x>0$ where it is energetically forbidden
to be.

Rewrite $\psi_{1}$as,
\begin{eqnarray*}
\psi_{1} & = & \frac{C}{2}\left(1+\frac{iK}{k_{1}}\right)e^{ik_{1}x}+\frac{C}{2}\left(1-\frac{iK}{k_{1}}\right)e^{-ik_{1}x}\\
 & = & C\left(\cos(k_{1}x)-\frac{K}{k_{1}}\sin(k_{1}x)\right)
\end{eqnarray*}
This is a standing wave solution from interference of incident wave
with the reflected wave

\begin{minipage}[t]{1\columnwidth}%
\noindent \begin{center}
\includegraphics{\string"Quantum Mechanics/wave and step\string".pdf}
\par\end{center}%
\end{minipage}

Due to full destructive interference between incident and reflected
wave because the amplitudes of the two waves are the same $|A|=|B|$.
In the region $x>0$, $|\psi|^{2}\propto e^{-2Kx}$. This falls by
a factor of $e\approx2.7$ in distance, $x=\frac{1}{2K}$ (penetration
distance).

If the particle was observed in this energetically forbidden region,
then energy would not be conserved and the kinetic energy would be
greater that the potential energy at this point. In order to observe
the particle in this position, $x>0$, then a spacial resolution of
$\Delta p\geq\frac{\hbar}{2\Delta x}\approx\hbar k$
\[
\Rightarrow\Delta E\approx\frac{\Delta p^{2}}{2m}\gtrsim\frac{\hbar}{2m}\frac{2m(V_{0}-E)}{\hbar^{2}}=V_{0}-E
\]
So if the measurement resolution of the location of the particle $x>0$,
is high enough to determine its position, then the resolution of the
energy measurement decreases so that $V_{0}-E$ cannot be measured
so the energy violation cannot be measured directly either.


\paragraph{Characteristic Quantum Result:}

$\psi$ appears to violate energy conservation, but quantum uncertainty
prohibits unphysical behavior being measured. But there are measurable
physical quantities.


\section{Solutions to TISE - Quantum Tunneling}

\marginpar{E+R pg 199-209}[-1cm]

\begin{minipage}[t]{1\columnwidth}%
\noindent \begin{center}
\includegraphics{\string"Quantum Mechanics/quantum tunelling\string".pdf}
\par\end{center}%
\end{minipage}

Consider the case when $V_{0}>E$. For a classical particle, incident
from the left, the region $x>0$ is energetically forbidden so 100\%
probability that it is reflected and 0\% transmitted.

Solve TISE for three regions and two boundaries. Write down free particle
solution for each separate region,
\begin{eqnarray*}
x\leq0,\quad\psi_{1} & = & Ae^{ikx}+Be^{-ikx},\qquad k=\frac{\sqrt{2mE}}{\hbar}\\
0<x<a,\quad\psi_{2} & = & Ce^{iKx}+De^{Kx},\qquad K=\frac{\sqrt{2m(V_{0}-E)}}{\hbar}\in\mathbb{R},>0\\
x\geq a,\quad\psi_{3} & = & Fe^{ikx},\qquad(\text{same }k\text{ since }V=0)
\end{eqnarray*}
Note: x cannot $\rightarrow\infty$ in $0\leq x\leq a$, so $D\neq0$
but $\psi_{3}$ has no $e^{-ikx}$term since no reflection beyond
$x=a$.


\paragraph{Boundary Conditions:}

Continuity $\psi,\,\psi`$ at $x=0$ and $x=a$,
\[
x=0,\qquad\psi_{1}(0)=\psi_{2}(0),\qquad\frac{d\psi_{1}(0)}{dx}=\frac{d\psi_{2}(0)}{dx}
\]
\[
x=a,\qquad\psi_{1}(a)=\psi_{2}(a),\qquad\frac{d\psi_{1}(a)}{dx}=\frac{d\psi_{2}(a)}{dx}
\]
The probability of the particle tunneling through the barrier to $x\rightarrow\infty$.
\[
\text{Transmission Coeffiecient}=T=\frac{k|F|^{2}}{k|A|^{2}}=\frac{|F|^{2}}{|A|^{2}}=\left|\frac{F}{A}\right|^{2}
\]
For a very high ($V_{0}\gg E$, k large) or very wide ($a\gg\frac{1}{k}$)
barrier ,
\[
T=\frac{16E}{V_{0}}\left(1-\frac{E}{V_{0}}\right)e^{-2Ka},\quad(ka\ll1)
\]
This is non zero, so there is a finite tunneling probability, though
this is generally a small number.

\begin{minipage}[t]{1\columnwidth}%
\noindent \begin{center}
\includegraphics{\string"Quantum Mechanics/wave and step wave\string".pdf}
\par\end{center}%
\end{minipage}

$|\psi_{1}|^{2}\neq0$ at nodes since $R<1$, incomplete interference.

$|\psi_{3}|^{2}=|Fe^{ikx}|^{2}=|F|^{2}=\text{constant}$

Can measure the particle in $x>a$ but cannot resolve the particle
inside the barrier.

For macroscopic objects, e.g.; $m=1\unit{kg},$ $v=1\unit{ms^{-1}}$,
$E=\frac{1}{2}mv^{2}=\frac{1}{2}J$, $V_{0}=1J$, $a=1\unit{cm}$,
\begin{eqnarray*}
\Rightarrow K & = & \frac{\sqrt{2m(V_{0}-E)}}{\hbar}=\frac{1}{\hbar}\\
 & = & 10^{34}\unit{m^{-1}}
\end{eqnarray*}
 
\begin{eqnarray*}
e^{ika} & \approx & e^{-(2\times10^{34})\times10^{-2}}=e^{-2\times10^{32}}\\
 & = & 10^{\frac{-2\times10^{32}}{\ln10}}\\
 & \approx & 0
\end{eqnarray*}
So for the macroscopic scale, we don't get quantum tunneling.

For the atomic scale, $m=9.1\times10^{-31}\unit{kg}$, $E=1\unit{eV}$,
$V_{0}=10\unit{eV}$, $a=1\text{\AA}=10^{-10}\unit{m}$
\[
\Rightarrow T=7\%
\]
So there is still a small but finite number for the atomic scale.

So far we have considered only free particles that are unbound ($\Delta x=\infty$),
\begin{itemize}
\item k depends on V, $k\propto\frac{1}{\lambda}\propto\sqrt{E-V_{0}}$.
\item any sudden change in potential (weather up or down) will cause both
the possibility of reflection and transmission.
\item the wave function inside an energetically forbidden region is a real
exponential $\neq0$ ($\psi\in\mathbb{R}\neq0$).
\item as a consequence, a particle can tunnel through a forbidden region
or barrier and be observed on the other side.
\end{itemize}

\section{Bound State Particles}

Consider a particle in a finite region $\Delta x<0$, this leads to
several new quantum phenomena. This will be considered for the simplified
one dimensional situation.


\subsection{Square Well Potential}

\marginpar{E+R pg 209-221}[-1cm]

\begin{minipage}[t]{1\columnwidth}%
\noindent \begin{center}
\includegraphics{\string"Quantum Mechanics/square well\string".pdf}
\par\end{center}%
\end{minipage}

If $E<V_{0}$, then the particle is bound as the energy is classically
forbidden outside the well and so the wave-function would quickly
tend to zero in this region. First consider the infinite square well,
where $V_{0}=\infty$ which is the simplest case. When $V=\infty$,
the only solution$\neq0$ is inside the well, i.e. the particle is
not found anywhere $|x|>\frac{a}{2}$.


\subsubsection{Solve TISE}

\[
\psi=A\sin(kx)+B\cos(kx)
\]



\paragraph{Boundary Conditions}
\begin{itemize}
\item $\psi$ continuous at edge of the well
\item $\psi=0$ outside the well
\end{itemize}
$\Rightarrow x=-\frac{a}{2}$
\begin{eqnarray}
0 & = & A\sin\left(-k\frac{a}{2}\right)+B\cos\left(-k\frac{a}{2}\right)\nonumber \\
 & = & -A\sin\left(k\frac{a}{2}\right)+B\cos\left(k\frac{a}{2}\right)
\end{eqnarray}
$\Rightarrow x=\frac{a}{2}$
\begin{equation}
0=A\sin\left(k\frac{a}{2}\right)+B\cos\left(k\frac{a}{2}\right)
\end{equation}
In this case, because $V=\infty$, $\psi'$ is not required to be
continuous at the edges of the well as it is unphysical to have an
infinite potential, so the normal rules do not apply.

\begin{eqnarray}
(10)+(11)\qquad B\cos\left(k\frac{a}{2}\right) & = & 0\\
(10)-(11)\qquad A\sin\left(k\frac{a}{2}\right) & = & 0
\end{eqnarray}
Try $A=B=0\;\Rightarrow\psi=0$ so no particle. So either

\begin{equation}
A=0\qquad\cos\left(k\frac{a}{2}\right)=0
\end{equation}
\noun{Or}
\begin{equation}
B=0\qquad\sin\left(k\frac{a}{2}\right)=0
\end{equation}
\[
(14)\rightarrow\psi=\cos\left(k\frac{a}{2}\right)\qquad k\frac{a}{2}=\frac{\pi}{2},\frac{3\pi}{2},\frac{5\pi}{2}...
\]
\[
(15)\rightarrow\psi=\sin\left(k\frac{a}{2}\right)\qquad k\frac{a}{2}=\pi,2\pi,3\pi...
\]
So we are left with two sets of solutions.
\begin{enumerate}
\item $\psi_{n}=B_{n}\cos(k_{n}x)\qquad\text{with }k_{n}=\frac{n\pi}{a},\qquad n=1,3,5...$
\item $\psi_{n}=A_{n}\sin(k_{n}x)\qquad\text{with }k_{n}=\frac{n\pi}{a},\qquad n=2,4,6...$
\end{enumerate}
So there is an infinite number of possible solutions. n is the quantum
number, labels solution (eigenstate). Corresponding energy eigenvalues;
\begin{eqnarray*}
E_{n} & = & \frac{\left(\hbar k_{n}\right)^{2}}{2m}\\
 & = & \frac{n^{2}\pi^{2}\hbar^{2}}{2ma^{2}}\qquad n=1,2,3...
\end{eqnarray*}
This means that the energy in snot a continuous variable, but has
distinct discrete values. So solving TISE has led to energy quantization,
only discrete energy values are allowed.

\begin{minipage}[t]{1\columnwidth}%
\noindent \begin{center}
\includegraphics{\string"Quantum Mechanics/square well energy levels\string".pdf}
\par\end{center}%
\end{minipage}


\subsubsection{Infinite Square Potential Well}

For an infinite well, there are infinite values of energy levels.
For square well $E_{n}\propto n^{2}$.

The ground-state (lowest energy of the system) has $n=1$,
\[
\Rightarrow E_{1}=\frac{\pi^{2}\hbar^{2}}{2ma^{2}}>0\qquad\text{zero point energy}
\]
For Heisenberg Uncertainty Principle, if $\Delta x<\infty$ (bound),
then $\Delta p>0$, so $p\neq0$ and thus $E\neq0$, zero point energy.
So any bound state has a non-zero lowest energy.


\subsubsection{Wave-functions}

\[
k_{n}=\frac{n\pi}{a},\qquad k=\frac{2\pi}{\lambda},\qquad a=\frac{n\lambda}{2}
\]
So there are an infinite number of \noun{half} wavelengths inside
the well, this just restates the boundary conditions that the wave-function
must got to zero at the edges of the well.

\begin{minipage}[t]{1\columnwidth}%
\noindent \begin{center}
\includegraphics{\string"Quantum Mechanics/square wells with waves\string".pdf}
\par\end{center}%
\end{minipage}

Two sets of solutions have different parities (symmetry about $x=0$)
\begin{eqnarray*}
\psi_{n} & = & B_{n}\cos(k_{n}x)\qquad\psi(x)=\psi(-x)\qquad\text{even parity for odd n}\\
\psi_{n} & = & A_{n}\sin(k_{n}x)\qquad\psi(x)=-\psi(-x)\qquad\text{odd parity for even n}
\end{eqnarray*}
$|\psi|^{2}$ is always even parity, given that the potential is symmetric
about $x=0$.


\subsection{Finite Square Potential Well}

\marginpar{E+R pg 209-214 and Appendix H}[-1cm]

Solve TISE for each of the three regions and two boundaries. Now the
wave-function is non-zero in the two energetically forbidden regions,
and there are two boundary conditions for each edge of the well;

\[
\psi_{1}(0)=\psi_{2}(0)
\]
and
\[
\frac{d\psi_{1}(0)}{dx}=\frac{d\psi_{2}(0)}{dx}
\]
Also $\psi\rightarrow0$ at either side as $x\rightarrow\infty$,
since $V_{0}<E$.

Again there are two sets of solutions with opposite parity.

\begin{minipage}[t]{1\columnwidth}%
\noindent \begin{center}
\includegraphics{\string"Quantum Mechanics/finite square wells with waves\string".pdf}
\par\end{center}%
\end{minipage}

As $V_{0}=\infty\rightarrow V_{0}=\text{finite}$, $\psi$ ``spreads
out'' into the walls of the well. So there is now a less than integer
number of half wavelengths inside the well. $\lambda$ has effectively
increased, so the energy has decreased. So the corresponding energy
levels in infinite and finite square wells will not be the same. Finite
well energy level $<$ infinite well energy level.

To get the energy eigenvalues, we need to solve the other half of
the boundary condition equations,
\begin{eqnarray*}
\tan\left(\frac{ka}{2}\right) & = & \frac{K}{k}\qquad\qquad\text{Even parity}\\
\cot\left(\frac{ka}{2}\right) & = & -\frac{K}{k}\qquad\qquad\text{Odd parity}
\end{eqnarray*}
There exists no analytical technique to solve these equations, so
we need a numerical method r graphical technique. Rewrite $E$, $V_{0}$,
\begin{eqnarray*}
\Rightarrow\tan\left(\frac{a}{\hbar}\sqrt{\frac{mE}{2}}\right) & = & \sqrt{\frac{V_{0}-E}{E}}\qquad\text{\textsc{Or}}\\
\Rightarrow-\cot\left(\frac{a}{\hbar}\sqrt{\frac{mE}{2}}\right) & = & \sqrt{\frac{V_{0}-E}{E}}
\end{eqnarray*}
For convenience, change the variables,
\begin{eqnarray*}
y & = & \frac{a}{\hbar}\sqrt{\frac{mE}{2}}=\frac{ka}{2}\\
\lambda & = & \frac{mV_{0}a^{2}}{2\hbar^{2}}=\text{constant}
\end{eqnarray*}
Note $\lambda>y^{2}$, when $V_{0}>E$,
\begin{eqnarray*}
\Rightarrow\tan(y) & = & \frac{\sqrt{\lambda-y^{2}}}{y}\qquad\text{\textsc{Or}}\\
\Rightarrow-\cot(y) & = & \frac{\sqrt{\lambda-y^{2}}}{y}
\end{eqnarray*}
Plot LHS and RHS as graphs as functions of $y$. The locations where
the two curves intersect gives the solutions of $y$. As $y$ is a
function of $E$, this gives $E$. When $\lambda\rightarrow\infty$
(so $V_{0}\rightarrow\infty$) the intersections tend to values of
$\frac{\tan}{\cot}$ at the poles (asymptotes), i.e. at $\pm\infty$.

\includepdf{\string"Quantum Mechanics/\string"Finite-well-solutions}

So
\begin{eqnarray*}
y & = & \frac{\pi}{2},\frac{3\pi}{2},\frac{5\pi}{2}...\qquad even\\
y & = & \pi,2\pi,3\pi...\qquad odd
\end{eqnarray*}
So $k=\frac{n\pi}{a}$ as expected from the previous infinite well
case. For $V_{0}$ finite, $\lambda$ is finite, the intersections
occur at smaller y-values, so the energy levels are lower for finite
well.

When $\lambda$ is finite, there are only a finite number of solutions
as there are only a finite number of intersections with $y^{2}<\lambda$
i.e. $E<V_{0}$. So there are an finite number of energy levels in
a finite well. For $E>V_{0}$, the particle would be unbound from
the well.

As $V_{0}$ is reduced, the higher energy levels are lost (as they
are unbound) but there is always at least one bound state so long
as $V_{0}>0$, ground state, $n=1$, with even parity.

\begin{minipage}[t]{1\columnwidth}%
\noindent \begin{center}
\includegraphics{\string"Quantum Mechanics/square well with unbound\string".pdf}
\par\end{center}%
\end{minipage}


\subsection{Simple Harmonic Oscillator Potential Well}

\marginpar{E+R pg 209-214 and Appendix I}[-1cm]

The SHO is an important potential and is physically more realistic
than the square well as it does not have any abrupt step in potential/
Classically, the simple harmonic oscillator is formed by displacing
mass $m$, by a distance $x$, from the equilibrium given there is
a restoring force proportional to the displacement, $-cx$,
\begin{eqnarray*}
F & = & m\ddot{x}=-cx\\
\ddot{x} & = & -\frac{c}{m}x=-\omega^{2}x\qquad\omega=\sqrt{\frac{c}{m}}\text{ (angular frequency)}
\end{eqnarray*}
In quantum mechanics, the force is linked to the displacement by,
\begin{eqnarray*}
F & = & -\frac{dV}{dx}=-cx\\
V & = & \int cxdx=\frac{1}{2}cx^{2}=\frac{1}{2}\omega^{2}x^{2}m
\end{eqnarray*}
This is a quadratic equation for the potential. So the shape of the
simple harmonic oscillator is a parabolic well.

\begin{minipage}[t]{1\columnwidth}%
\noindent \begin{center}
\includegraphics[scale=0.8]{\string"Quantum Mechanics/parabolic well\string".pdf}
\par\end{center}%
\end{minipage}

Again, this is an infinite and symmetric well, so we would expect
an infinite number of solutions with energy quantization, and that
the eigenfunctions have even or odd parity. $k$ (hence $\frac{1}{\lambda}$)
varies across the well as $\sqrt{E-V}$ changes. The classical oscillator
would be limited to the width of the well at the given energy level,
\begin{eqnarray*}
E=V & = & \frac{1}{2}m\omega^{2}x^{2}\\
\Rightarrow x_{m} & = & \pm\sqrt{\frac{2E}{m\omega^{2}}}
\end{eqnarray*}


Quantum effects, however, mean that the quantum particle is not confined
strictly to the potential well, but will be able to penetrate slightly
into the wall. But the wave-function tends to zero rapidly as $|x|\rightarrow\infty$.
TISE then,
\[
-\frac{\hbar^{2}}{2m}\frac{d^{2}\psi}{dx^{2}}+\frac{1}{2}m\omega^{2}x^{2}=E\psi
\]
In this case, there are no regions of flat potential (free particle
solution) and so the full solution of TISE must be performed. See
appendix I in E+R for solution. Change in spatial variable to 
\[
u=x\sqrt{\frac{m\omega}{\hbar}}
\]
then the eigenfunctions are,
\[
\psi_{n}(u)=h_{n}(u)E^{-\frac{u^{2}}{2}}
\]
where $h_{n}$ is a Hermite polynomial ($n=0,1,2,3...$). The first
few (unnormalized) solutions are,
\begin{eqnarray*}
\psi_{0}(u) & = & e^{-\frac{u^{2}}{2}}\\
\psi_{1}(u) & = & ue^{-\frac{u^{2}}{2}}\\
\psi_{2}(u) & = & (2u^{2}-1)e^{-\frac{u^{2}}{2}}
\end{eqnarray*}
This satisfies the infinite number of bound states of even and odd
parity penetrating into the potential wall. Check by substituting
into TISE,
\begin{eqnarray*}
\psi_{0} & = & e^{-\frac{u^{2}}{2}}=e^{-\frac{m\omega x^{2}}{2\hbar}}\\
\frac{d\psi_{0}}{dx} & = & -\frac{m\omega x}{\hbar}e^{-\frac{m\omega x^{2}}{2\hbar}}\\
\frac{d^{2}\psi_{0}}{dx^{2}} & = & -\frac{m\omega}{\hbar}e^{-\frac{m\omega x^{2}}{2\hbar}}-\frac{m^{2}\omega^{2}x^{2}}{\hbar^{2}}e^{-\frac{m\omega x^{2}}{2\hbar}}\\
 & = & \left(-\frac{m\omega}{\hbar}-\frac{m^{2}\omega^{2}x^{2}}{\hbar^{2}}\right)e^{-\frac{m\omega x^{2}}{2\hbar}}
\end{eqnarray*}
\begin{eqnarray*}
\text{TISE}\Rightarrow-\frac{\hbar^{2}}{2m}\left(-\frac{m\omega}{\hbar}-\frac{m^{2}\omega^{2}x^{2}}{\hbar^{2}}\right)e^{-\frac{m\omega x^{2}}{2\hbar}}\psi_{0}+\frac{1}{2}m\omega^{2}x^{2}\psi_{0} & = & E_{0}\psi_{0}\\
\frac{1}{2}\hbar\omega\psi_{0} & = & E_{0}\psi_{0}
\end{eqnarray*}
So the ground state of the energy of the simple harmonic oscillator
is given by,
\[
E=\frac{1}{2}\hbar\omega
\]
this is the zero point energy.
\begin{eqnarray*}
n=0,\qquad E_{0} & = & \frac{1}{2}\hbar\omega\\
n=1,\qquad E_{1} & = & \frac{3}{2}\hbar\omega\\
n=2,\qquad E_{2} & = & \frac{5}{2}\hbar\omega\\
 & \vdots\\
n=n,\qquad E_{n} & = & \left(n+\frac{1}{2}\right)\hbar\omega
\end{eqnarray*}
for the simple harmonic oscillator. The energy levels are now evenly
spaced with a linear dependence on$n$, with the spacing being $\hbar\omega$,
\[
E_{n+1}-E_{n}=\hbar\omega
\]
which is the energy of the photon. The probability function, $|\psi|^{2}$,
varies with location across the well. Classically, the PDF is smooth
with peaks at the edges of the range. At large values of $n$, the
individual peaks and troughs cannot be resolved at the classical situation
is reached. 
\[
\text{So as }n\rightarrow\infty,\text{ quantum}\rightarrow\text{classical}
\]
This is an example of the \noun{Correspondence Principle}.

\includepdf[pages=-]{\string"Quantum Mechanics/\string"sho-wavefunctions_qm-clas}


\part{Time Dependence, Superposition and Measurement}


\section{Time Dependence}

\marginpar{E+R pg 166-167}[-1cm]

If $\psi_{n}(x)$ is a solution of TISE (spacial wave-function) then
the full wave-function,
\[
\Psi(x,t)=\psi_{n}(x)e^{-\frac{E_{n}t}{\hbar}}
\]
where $E_{n}$ is the energy eigenvalue. This is the solution of the
full SE,
\begin{eqnarray*}
|\Psi|^{2} & = & \left(\psi_{n}^{*}e^{i\frac{E_{n}t}{\hbar}}\right)\left(\psi_{n}e^{-i\frac{E_{n}t}{\hbar}}\right)\\
 & = & \psi_{n}^{*}\psi_{n}=|\psi|^{2}
\end{eqnarray*}
which has no time dependence. This is true when the wave-function
is an eigenfunction of the Hamiltonian.

Consider the mixed state,
\begin{eqnarray*}
\Phi(x,t) & = & \Psi_{m}(x,t)+\Psi_{n}(x,t)\\
|\Phi|^{2} & = & |\Psi_{m}+\Psi_{n}|^{2}\\
 & = & \Psi_{m}^{*}\Psi_{m}+\Psi_{n}^{*}\Psi_{n}+\underbrace{\Psi_{m}^{*}\Psi_{n}+\Psi_{n}^{*}\Psi_{m}}_{\text{interferance terms}}\\
 & = & |\Psi_{m}|^{2}+|\Psi_{n}|^{2}+\Psi_{m}^{*}\Psi_{n}e^{-i\frac{\left(E_{n}-E_{m}\right)t}{\hbar}}+\Psi_{n}^{*}\Psi_{m}e^{-i\frac{\left(E_{n}-E_{m}\right)t}{\hbar}}\\
 & = & |\Psi_{m}|^{2}+|\Psi_{n}|^{2}+\mathbb{R}\left\{ \Psi_{m}^{*}\Psi_{n}e^{-i\frac{\left(E_{n}-E_{m}\right)t}{\hbar}}\right\} \\
 & = & |\Psi_{m}|^{2}+|\Psi_{n}|^{2}+2\Psi_{m}^{*}\Psi_{n}\cos\left(\frac{\Delta Et}{\hbar}\right)
\end{eqnarray*}
$\Delta E=E_{n}-E_{m}$, and assuming that $\Psi_{m}^{*}\Psi_{m}$
is real, if its not then also term $\propto\sin\left(\frac{\Delta Et}{\hbar}\right)$.
So the probability density oscillates between states with an angular
frequency,
\[
\omega=\frac{\Delta E}{\hbar}
\]
$\Phi(x,t)=\Psi_{m}(x,t)+\Psi_{n}(x,t)$ is called a superposition
states. If both $\Psi_{m}$ and $\Psi_{n}$ are solutions of the full
SE, the linearity means that $\Phi$ must also be a solution for the
same potential function, $V$. But
\[
\phi(x)=\psi_{m}(x)+\psi_{n}(x)
\]
is \noun{not }a solution of TISE because 
\begin{eqnarray*}
\hat{\mathcal{H}}\phi & = & \hat{\mathcal{H}}\left(\psi_{m}+\psi_{n}\right)\\
 & = & E_{m}\psi_{m}+E_{n}\neq\psi_{n}E\left(\psi_{m}+\psi_{n}\right)
\end{eqnarray*}
since $E_{m}\neq E_{n}$. $\Phi(x,t)$ is not an eigenfunction of
$\hat{\mathcal{H}}$, so does not have a definite value of E.


\section{Superposition }

\marginpar{G ch 4+6}[-1cm]

If $\Phi_{i}(x,t)$ ($i=1,2,3,\ldots$) form a complete set of solutions
of SE, (for certain $V(x,t)$) then the general solution can be expressed
as 
\[
\Phi(x,t)=\sum_{i=1}^{n}C_{i}\Psi_{i}(x,t)
\]
with $C_{i}$ ($i=1,2,3,\ldots$) being a constant coefficient which
are in general complex. Any valid wave-function can be written in
this way. In general, $\Phi$ is not an eigenfunction of $\hat{\mathcal{H}}$
of other operators, so have to have no definite value for the energy
or other observables. So what is the result of a measurement of energy
etc?


\subsubsection{More on Eigenfunctions and Eigenvalues}

If $\hat{A}\psi_{i}=a_{i}\psi_{i}$ for the operator $\hat{A}$ corresponding
to the observable $A$, then the value of this observable is given
by the eigenvalue $a_{i}$ for the particle in state $\psi_{i}$.
State $\psi_{i}$ has a well defined value of $A$. A complete set
of eigenfunctions $a_{i}$ ($i=1,2,3,\ldots$), gives all possible
results of measuring this observable, for this system, for this state.
Other values are not allowed (quantization).

$\therefore$ all eigenvalues, $a_{i}$, must be real.


\subsection{Hermitian Operators}
\begin{description}
\item [{Definition}] Hermitian operators satisfy the conditions
\[
\int_{-\infty}^{\infty}\psi_{b}^{*}\hat{\mathcal{O}}\psi_{a}dx=\int_{-\infty}^{\infty}\psi_{a}(\hat{\mathcal{O}}\psi_{b})^{*}dx
\]

\end{description}
This is in one dimension, but is the same for the general situation
in all space, for any two wave-functions $\psi_{a}$ and $\psi_{b}$.

Suppose $\psi_{a}$ is an eigenfunction: $\hat{\mathcal{O}}\psi_{a}=a\psi_{a}$,
then
\begin{eqnarray*}
\int\psi_{a}^{*}\hat{\mathcal{O}}\psi_{a}dx & = & \int\psi_{a}(\hat{\mathcal{O}}\psi_{a})^{*}dx\\
\int\psi_{a}^{*}a\psi_{a}dx & = & \int\psi_{a}(a\psi_{a})^{*}dx\\
a\int\psi_{a}^{*}\psi_{a}dx & = & a^{*}\int\psi_{a}\psi_{a}^{*}dx\\
\Rightarrow a & = & a^{*}\qquad\therefore a\in\mathbb{R}
\end{eqnarray*}
So Hermitian operators produce eigenvalues that are always real. This
condition is required as they produce a physical, measurable property.
Suppose also that $\psi_{b}$ is another eigenfunction $\hat{\mathcal{O}}\psi_{b}=b\psi_{b}$,
\begin{eqnarray*}
\int\psi_{b}^{*}\hat{\mathcal{O}}\psi_{a}dx & = & \int\psi_{a}(\hat{\mathcal{O}}\psi_{b})^{*}dx\\
\int\psi_{b}^{*}a\psi_{a}dx & = & \int\psi_{a}(b\psi_{b})^{*}dx\\
a\int\psi_{b}^{*}\psi_{a}dx & = & b\int\psi_{a}\psi_{b}^{*}dx\\
(a-b)\int\psi_{b}^{*}\psi_{a}dx & = & 0\\
\text{If }a\neq b,\qquad\int\psi_{b}^{*}\psi_{a}dx & = & 0\\
\text{If }a=b,\qquad\int\psi_{a}^{*}\psi_{a}dx & = & 1
\end{eqnarray*}
In general,
\[
\int_{-\infty}^{\infty}\psi_{b}^{*}\psi_{a}=\left\{ \begin{array}{cc}
0 & \text{if }a\neq b\\
1 & \text{if }a=b
\end{array}\right\} =\delta_{ab}\text{ (Kronecker Delta)}
\]
Therefore Hermitian operators are orthogonal. In quantum mechanics,
Hermitian operators represent physical observables, real eigenvalues
give the measured value of the observable, and the orthogonal eigenfunctions
present a basis to write any wave-function $\Phi$ as superposition,
\[
\Phi=\sum_{i=1}^{n}C_{i}\Psi_{i}
\]



\subsection{Expectation Value for Superposition States}

\begin{eqnarray*}
\phi & = & \sum_{i=1}^{n}C_{i}\Psi_{i}\\
<o> & = & \int\psi^{*}\hat{\mathcal{O}}\psi dx=\int\phi^{*}\hat{\mathcal{O}}\phi dx\\
 & = & \int\left(\sum_{i=1}^{n}C_{i}\psi_{i}\right)^{*}\hat{\mathcal{O}}\left(\sum_{j=1}^{n}C_{j}\psi_{j}\right)dx\\
 & = & \int\sum_{i=1}^{n}C_{i}^{*}\psi_{i}^{*}\sum_{j=1}^{n}C_{j}a_{j}\psi_{j}dx\\
 & = & \sum_{i=1}^{n}\sum_{j=1}^{n}C_{i}^{*}C_{j}a_{j}\int\psi_{j}^{*}\psi_{i}dx\\
 & = & \sum_{i=1}^{n}\sum_{j=1}^{n}C_{i}^{*}C_{j}a_{j}\delta_{ij}\\
 & = & \sum_{i=j=1}^{n}C_{i}^{*}C_{j}a_{j}=\sum_{j=1}^{n}|C_{j}|^{2}a_{j}
\end{eqnarray*}
When $i\neq j$, the values of $\delta_{ij}=0$ so the sum collapses
for all values other than when $i=j$.

Identify $|C^{2}|=\text{probability of measuring observable}$ $o$
to have a value of $a_{j}$ for the state $\phi$. Eigenfunction value
$<o>=\text{weighted average}$.
\begin{eqnarray*}
 & = & \sum_{i=1}^{n}\left[\begin{array}{cc}
\text{probability of} & \times\;\text{ value}\\
\text{measuring this value}
\end{array}\right]\\
 & = & \sum_{i=1}^{n}\left[|C_{i}|^{2}\times a_{i}\right]
\end{eqnarray*}
Also note that 
\[
\int_{-\infty}^{\infty}\phi^{*}\phi dx=\sum_{i=1}^{n}|C_{i}|^{2}=1=\text{total probability}
\]


\uline{\noun{Ex}}

$\phi=\sqrt{\frac{2}{a}}\cos\left(\frac{\pi x}{a}\right)$ is a free
particle normalized over the space $-\frac{a}{2}$ to $\frac{a}{2}$.
$\phi$ is an eigenfunction of kinetic energy,
\[
\hat{E_{k}}=\frac{\hat{P}}{2m}\phi=\frac{\pi^{2}\hbar^{2}}{2ma^{2}}\phi
\]
But this state, $\phi$, is not an eigenfunction of momentum since
the cosine becomes sine and the original wave-function is lost,
\[
\hat{P}=-i\hbar\frac{d}{dx}
\]
Write $\phi$ as the superposition in terms of $\hat{P}$ eigenfunction
\begin{eqnarray*}
\psi_{+} & = & \frac{1}{\sqrt{a}}e^{i\pi x},\qquad\hat{P}\psi_{+}=\frac{\pi\hbar}{a}\psi_{+}\\
\psi_{-} & = & \frac{1}{\sqrt{a}}e^{-i\pi x},\qquad\hat{P}\psi_{-}=\frac{-\pi\hbar}{a}\psi_{-}
\end{eqnarray*}
these are also normalized over the space $-\frac{a}{2}$ to $\frac{a}{2}$.
\begin{eqnarray*}
\Rightarrow\phi & = & \sqrt{\frac{2}{a}}\left(\frac{e^{i\frac{\pi x}{2}}+e^{-i\frac{\pi x}{2}}}{2}\right)\\
 & = & \frac{1}{\sqrt{2}}\psi_{+}+\frac{1}{\sqrt{2}}\psi_{-}=C_{+}\psi_{+}+C_{-}\psi_{-}\qquad C_{+}=C_{-}=\frac{1}{\sqrt{2}}
\end{eqnarray*}
For the state $\phi$, the probability measurement of the momentum
value
\begin{eqnarray*}
p_{+} & \Rightarrow & |C_{+}|^{2}=\frac{1}{2}\\
p_{-} & \Rightarrow & |C_{-}|^{2}=\frac{1}{2}
\end{eqnarray*}
So the particle is equally likely to be moving along the $\pm$ x-axis
with $|p|=\frac{\pi\hbar}{a}$, $E_{k}=\frac{\pi^{2}\hbar^{2}}{2ma^{2}}$
\[
<p>=|C_{+}|^{2}p_{+}+|C_{-}|^{2}p_{-}=0
\]



\subsubsection{Calculating Coefficients $C_{i}$ in General}

\[
\phi=\sum_{i=1}^{n}C_{i}\psi_{i}
\]
\begin{eqnarray*}
\int\psi_{j}^{*}\phi dx & = & \sum_{i=1}^{n}C_{i}\int\psi_{j}^{*}\psi_{i}dx\\
 & = & \sum_{i=1}^{n}C_{i}\delta_{ij}=C_{j}\\
C_{j} & = & \int_{-\infty}^{\infty}\psi_{j}^{*}\phi dx\qquad\text{(overlap integral)}
\end{eqnarray*}
So for a given state $\phi$, eigenfunction $\psi_{i}$ ($i=1,2,3,\ldots n$),
the coefficients $C_{i}$ ($i=1,2,3,\ldots n$) can be calculated.
Hence the probabilities $|C_{i}|^{2}$ ($i=1,2,3,\ldots n$) of measuring
the observable, $o$, having the value $a_{i}$ for the state $\phi_{i}$
where 
\[
\hat{\mathcal{O}}\psi_{i}=a_{i}\psi_{i}
\]



\section{Measurement and Collapse of Wave-functions}

Consider repeat measurements on many particles, each in the same state
described by $\phi$. We will get different results for each measurement
corresponding to the different eigenvalues, $a_{1},a_{2},a_{3}\ldots a_{n}$
with probabilities $|C_{1}|^{2},|C|_{2}^{2},|C_{3}|^{2}\ldots|C|_{n}^{2}$.
Quantum mechanics does not predict definite results of single measurements,
but the average is 
\[
<\mathcal{O}>=\sum_{i=1}^{n}|C_{i}|^{2}a_{i}
\]
This can be predicted and compared to experimental measurements. If
instead a single particle is measured, there will be a definite result,~$a_{j}$,
probability is given by $|C_{j}|^{2}$ for the observable, $\mathcal{O}$.
Suppose, the, the same observable is measured again immediately after
the first. We would expect the get the same result for the same particle.
We would expect the same value for $\mathcal{O}$ with a probability
of $1$ (certainty).

The particle has some definite value, $a_{j}$, for the observable
$\mathcal{O}$. So now the particle must be described by a different
wave-function (not $\phi$) so that it is an eigenfunction of the
operator for the observable $\mathcal{O}$. After the first measurement,
the particle is in an eigenstate of $\mathcal{O}$ with eigenvalue,
$a_{j}$, i.e. the wave-function is $\psi_{j}$. This process is called
the collapse of the wave-function from $\phi\rightarrow\psi_{j}$.
Collapse is discontinuous and irreversible process and cause the loss
of all history of the previous state $\phi$.

\uline{\noun{Ex}}

Initial state $\phi=\sqrt{\frac{2}{a}}\cos\left(\frac{\pi x}{a}\right)$,
which is an eigenstate of $E_{k}$ but not $p$), where the momentum
eigenstates are $=\frac{1}{\sqrt{2}}\psi_{+}+\frac{1}{\sqrt{2}}\psi_{-}$
\[
\psi_{\pm}=\frac{1}{\sqrt{a}}e^{\pm\frac{i\pi x}{a}}
\]
Eigenvalues $P_{\pm}=\pm\frac{\pi\hbar}{a}$

Measure the momentum for this particle gives the result $p_{+}=+\frac{\pi\hbar}{a}$
(probability$=\frac{1}{2}$) When the wave-function collapses, $\phi\rightarrow\psi_{+}=\frac{1}{\sqrt{a}}e^{\frac{i\pi x}{a}}$,
\[
\psi_{+}=\left(\cos\left(\frac{\pi x}{a}\right)-i\sin\left(\frac{\pi x}{a}\right)\right)\frac{1}{\sqrt{a}}\neq\phi
\]
The collapse of the wave-function does not follow SE. It is a separate
postulate of QM. Interpretation of this postulate is still controversial
an leads to a purely philosophical debate,
\begin{enumerate}
\item Is superposition time independent of states or just our knowledge
of it?

\begin{itemize}
\item Can observe quantum interference easily which leads to the superposition
of the real state.
\end{itemize}
\item Is the result of a measurement on superposition unknown or actually
indeterminate beforehand?

\begin{itemize}
\item Quantum mechanics says indeterminate. Experimental verification by
Alan Aspect and from ``Bell's inequalities''.
\end{itemize}
\item What is ``measurement''? and when does the collapse actually ``occur''?

\begin{itemize}
\item Schrödinger's Cat thought experiment lets us consider the situation

\begin{itemize}
\item Unpredictable, unstable nuclear decay can be detected. This causes
the death of the cat. So the fate of the cat is determined by the
state of the nucleus. This happens in a box whereby the experimenters
do not know the state of the system. So before the box is opened,
the cat is in a superposition of the alive and dead states.
\end{itemize}
\item Common usage is the Copenhagen Interpretation which basically says
``ignore the philosophy and use it as a tool''.
\end{itemize}
\end{enumerate}

\section{Commutation}

A wave-function can be an eigenfunction of more that one operator,
e.g.
\[
\Psi(x,t)=Ae^{i(kx-\omega t)}
\]
is an eigenfunction of momentum, $\hat{P}=-i\hbar\frac{\partial}{\partial x}$,
with eigenvalue $p=\hbar k$, and of energy, $\hat{E}=i\hbar\frac{\partial}{\partial t}$
with eigenvalue $E=\hbar\omega$. So we could know precisely the value
of the momentum and the energy at the same time, without uncertainty.
But some combinations of observables cannot be known simultaneously,
e.g.
\[
\Delta x\Delta p\geq\frac{\hbar}{2}
\]
which is one form of the Heisenberg Uncertainty Principle. Commutation
refers to the order in which operators or functions are acted. Most
numbers, functions etc commute,
\begin{eqnarray*}
2\times3 & = & 3\times2\\
xy & = & yx\\
 & \vdots
\end{eqnarray*}
But in general, this is not true of operators, they don't all commute,
\[
x\frac{d}{dx}\psi\neq\frac{d}{dx}(x\psi)
\]
In general 
\begin{eqnarray*}
\hat{A}\hat{B}\psi & \neq & \hat{B}\hat{A}\psi\\
\left(\hat{A}\hat{B}-\hat{B}\hat{A}\right)\psi & \neq & 0
\end{eqnarray*}
$\left(\hat{A}\hat{B}-\hat{B}\hat{A}\right)$ is called the commutator
of operators $A$ and $B$ and is rewritten as $\left[\hat{A},\hat{B}\right]\psi\neq0$.
For the general situation, with wave-function $\psi$,$\left[\hat{A},\hat{B}\right]\neq0$.
$\hat{A}\text{ and }\hat{B}$ do not commute. For most pairs of operators
this is true.


\subsection{Simultaneous Eigenfunctions}

$\hat{A}\text{ and }\hat{B}$ are said to share a complete set of
simultaneous eigenfunctions if,
\begin{eqnarray*}
\hat{A}\psi_{i} & = & a_{i}\psi_{i}\\
\hat{B}\psi_{i} & = & b_{i}\psi_{i}
\end{eqnarray*}
for all $i=1,2,3\ldots n$. Operators that have this property, that
share simultaneous eigenfunctions, do commute,
\begin{eqnarray*}
\left[\hat{A},\hat{B}\right] & = & 0\\
\left(\hat{A}\hat{B}-\hat{B}\hat{A}\right)\psi_{i} & = & \left[\hat{A},\hat{B}\right]\psi_{i}\\
 & = & \hat{A}b_{i}\psi_{i}-\hat{B}a_{i}\psi_{i}\\
 & = & 0
\end{eqnarray*}
since numbers commute. This is true for the general function $\phi=\sum_{i=1}^{n}C_{i}\psi_{i}$
\[
\left[\hat{A},\hat{B}\right]\phi=\sum_{i=1}^{n}C_{i}\left[\hat{A},\hat{B}\right]\psi_{i}=0\Rightarrow\left[\hat{A},\hat{B}\right]=0
\]
Therefore $\hat{A}$ and $\hat{B}$ commute. The converse argument
is also true, any two operators that commute also share simultaneous
eigenfunction. This can be proved.

This result has a consequence in terms of measuring the observables.
The initial state
\[
\phi=\sum_{i=1}^{n}C_{i}\psi_{i}
\]
Measuring the observable A gives a result, $a_{j}$, with the probability
$|C_{J}|^{2}$. This causes the state to collapse to $\psi_{j}$.
now take a second measurement of the observable $B$. This now has
the definite value $b_{j}$, with a probability of $1$ because $\psi_{j}$
is an eigenfunction of $B$. This second measurement will have no
effect on the wave-function $\psi_{j}$ and the state is left unchanged
for the second measurement. So if $\hat{A}$ and $\hat{B}$ commute,
then $\hat{A}$ and $\hat{B}$ share a complete set of eigenfunctions
and the precise values of the observables $a$ and $b$ can be known
precisely at the same time. 

Consider the case where $\hat{A}$ and $\hat{B}$ do not commute,
e.g.
\begin{eqnarray*}
\left[\hat{x},\hat{P}\right]\psi & = & \left[x,-i\hbar\frac{\partial}{\partial x}\right]\psi\\
 & = & i\hbar\left(x\frac{\partial}{\partial x}\psi-\frac{\partial}{\partial}(x\psi)\right)\\
 & = & i\hbar\left(x\frac{\partial}{\partial x}\psi-x\frac{\partial}{\partial}\psi-\psi\right)\\
 & = & i\hbar\psi\neq0
\end{eqnarray*}
This result is true for any function, so $\left[\hat{x},\hat{P}\right]=i\hbar\neq0$
and so $\hat{x}$ and $\hat{P}$ do not commute. So $\hat{x}$ and
$\hat{P}$ do not share a complete set of eigenfunctions so we cannot
know the value of $x$ and $p$ precisely at the same time. This leads
to an uncertainty product relationship..


\subsubsection{Root Mean Squared}

Define the statistical root mean squared of the uncertainty on the
observable $A$,
\begin{eqnarray*}
(\Delta A)^{2} & = & \bigl\langle(A-\bigl\langle A\bigr\rangle)^{2}\bigr\rangle\\
 & = & \bigl\langle A^{2}\bigr\rangle-\bigl\langle A\bigr\rangle^{2}
\end{eqnarray*}
It can be shown that
\[
(\Delta A)^{2}(\Delta B)^{2}\geq\frac{1}{4}\left|\bigl\langle\left[\hat{A},\hat{B}\right]\bigr\rangle\right|^{2}
\]
e.g. 
\begin{eqnarray*}
(\Delta x)^{2}(\Delta p)^{2} & = & \frac{1}{4}|i\hbar|^{2}=\frac{\hbar^{2}}{4}\\
\Delta x\Delta p & = & \frac{\hbar}{2}
\end{eqnarray*}
This is the Heisenberg Uncertainty Principle. It is generally true
that for any two pairs of operators that do not commute, there exists
an uncertainty product, e.g.
\begin{eqnarray*}
\left[\hat{E},\hat{t}\right]\psi & = & \left[i\hbar\frac{\partial}{\partial t},t\right]\psi=i\hbar\psi\\
\Delta E\Delta t & \geq & \frac{\hbar}{2}
\end{eqnarray*}
This relationship relates to the decay time interval $\Delta t$.
The uncertainty in the energy released by
\[
\Delta E\geq\frac{\hbar}{2\Delta t}
\]


\uline{\noun{Ex}}

Spectral line width - consider an excited atom which decays and emits
a gamma photon $\gamma$ with energy $E$. The lifetime, $\tau$,
is related to

\[
N(t)=N(0)e^{-\frac{t}{\tau}}
\]
Let $\Delta t=\tau$, then $\Delta E\geq\frac{\hbar}{2\tau}$. If
$\tau=10^{-7}\unit{s}$, then $\Delta E\geq\frac{\hbar}{2\times10^{-7}}\approx3\times10^{-9}\unit{eV}$

This gives us the intrinsic quantum limit to the resolution on the
line width , $\frac{\Delta E}{E}$. The practice experimental limit
is usually much greater than this but this gives the absolute minimum
resolution for a perfect experiment. 

\uline{\noun{Ex}}

Particle decay width - rho meson
\[
\rho\rightarrow\pi+\pi
\]
(fast decay governed by the strong force)> The $\rho$ mass can be
reconstructed from the energy of the two pions, this gives the invariant
mass of the rho meson.

\begin{minipage}[t]{1\columnwidth}%
\noindent \begin{center}
\includegraphics{\string"Quantum Mechanics/rho meson\string".pdf}
\par\end{center}%
\end{minipage}

\begin{eqnarray*}
\Delta E & = & 75\unitfrac{MeV}{c^{2}}\\
\Rightarrow\Delta t\Delta E & \geq & \frac{\hbar}{2}\\
\Rightarrow\tau_{\rho} & = & \Delta t\approx\frac{\hbar}{2\Delta E}\approx4\times10^{-24}\unit{s}
\end{eqnarray*}
This is the same order of magnitude as the time taken for light to
travel a distance equal to the radius of the rho meson.


\part{3-Dimensional Quantum Mechanics}


\section{Some Obvious Generalizations}

So far we have considered only the one dimensional simplification
of the Schrödinger Equation. In three dimensions, $\underbar{r}=(x,y,z)$
or $(r,\theta,\phi)$. This leads to some emergence of new quantum
phenomena.


\subsection{Operators}


\subsubsection{Momentum Operator}

\[
\hat{p_{x}}=-i\hbar\frac{\partial}{\partial x},\;\hat{p_{y}}=-i\hbar\frac{\partial}{\partial y},\;\hat{p_{z}}=-i\hbar\frac{\partial}{\partial z}
\]
Define $\underbar{\ensuremath{\hat{P}}}=(\hat{P_{x}},\hat{P_{y}},\hat{P_{z}})=-i\hbar\underline{\nabla}$


\subsubsection{Kinetic Energy Operator}

\[
\frac{\hat{P}^{2}}{2m}=-\frac{\hbar^{2}}{2m}\nabla^{2}
\]
Hence in three dimensional the SE becomes
\[
-\frac{\hbar^{2}}{2m}\nabla^{2}+V\Psi=i\hbar\frac{\partial\Psi}{\partial t}
\]
This is the full time dependent case where $V$ and $\Psi$ are functions
of $(x,y,z,t)$. If $V=V(\underline{r})$, independent of $t$, the
the three dimensional TISE becomes 
\[
-\frac{\hbar^{2}}{2m}\nabla^{2}\psi(\underline{r})+V(\underline{r})\psi(\underline{r})=E\psi(\underline{r})
\]
with $\Psi(\underline{r},t)=\psi(\underline{r})e^{-\frac{iEt}{\hbar}}$


\subsubsection{Momentum Eigenfunctions}

\[
\hat{P_{y}}e^{ik_{y}y}=\hbar k_{y}e^{ik_{y}y}
\]
In three dimensions, momentum eigenfunction is the product of the
three one dimensional forms,

\begin{eqnarray*}
p & = & Ae^{ik_{x}x}e^{ik_{y}y}e^{ik_{z}z}\\
 & = & Ae^{i(k_{x}x+k_{y}y+k_{z}z)}\\
 & = & Ae^{i(k\cdot\underline{r})}
\end{eqnarray*}
where $\underline{k}=(k_{x},k_{y},k_{z})$ which is called the wave
vector and is the equivalent of the wave number from one dimension.
\[
\underline{\hat{P}}e^{ik\cdot\underline{r}}=\hbar\underline{k}e^{ik\cdot\underline{r}}
\]
So the eigenvalue is $\underline{P}=\hbar k$, which is itself a vector.


\subsection{Uncertainty Relationships in 3D}

\[
\Delta x\Delta p\geq\frac{\hbar}{2}
\]
This applies in each dimension separately,
\begin{eqnarray*}
\left[\hat{y},\hat{P_{y}}\right] & = & i\hbar,\qquad\left[\hat{z},\hat{P_{z}}\right]=i\hbar\\
\Delta y\Delta p_{y} & \geq & \frac{\hbar}{2},\qquad\Delta z\Delta p_{z}\geq\frac{\hbar}{2}
\end{eqnarray*}


But now there is the possibility of interaction across dimensions.
These same patters do not hold across dimensions, and so there is
no uncertainty product,
\begin{eqnarray*}
\left[\hat{x},\hat{P_{y}}\right] & = & 0\\
\left[\hat{z},\hat{P_{x}}\right] & = & 0\\
 & \vdots
\end{eqnarray*}
So we can know the position and momentum along perpendicular directions
at the same time, just not parallel directions.


\section{3D Infinite Square Potential Well - Particle in a Box}

Consider a particle confined to a three dimensional right rectangular
box,

\begin{minipage}[t]{1\columnwidth}%
\noindent \begin{center}
\includegraphics{\string"Quantum Mechanics/particle in a box\string".pdf}
\par\end{center}%
\end{minipage}

Outside the box, $V=\infty\rightarrow\psi(\underline{r})=0$ everywhere

Inside the box, $\psi(\underline{r})$ is a solution of 3D TISE,
\begin{eqnarray*}
-\frac{\hbar^{2}}{2m}\left(\frac{\partial^{2}\psi}{\partial x^{2}}+\frac{\partial^{2}\psi}{\partial y^{2}}+\frac{\partial^{2}\psi}{\partial z^{2}}\right) & = & E\psi\\
-\frac{\hbar^{2}}{2m}\nabla^{2} & = & E\psi
\end{eqnarray*}
Solve using \noun{Separation of Variables. }Try solution of the form
\[
\psi(x,y,z)=X(x)Y(y)Z(z)
\]
Substitute this guess into TISE, so that the partial differential
equation becomes three ordinary differential equations,

\begin{eqnarray*}
-\frac{\hbar^{2}}{2m}\frac{d^{2}X}{dx^{2}} & = & E_{x}X\qquad0\leq x\leq a\\
-\frac{\hbar^{2}}{2m}\frac{d^{2}Y}{dy^{2}} & = & E_{y}Y\qquad0\leq y\leq b\\
-\frac{\hbar^{2}}{2m}\frac{d^{2}Z}{dz^{2}} & = & E_{z}Z\qquad0\leq z\leq c
\end{eqnarray*}
where $E=E_{x}+E_{y}+E_{z}$. Each of these is the same as for the
one dimensional well, so the solutions are know. Apply boundary conditions,
which gives the solutions,
\begin{eqnarray*}
X(x) & = & A_{x}\sin\left(\frac{n_{x}\pi x}{a}\right)\qquad(n_{x}=1,2,3\ldots)\qquad E_{x}=\frac{n_{x}^{2}\pi^{2}\hbar^{2}}{2ma^{2}}\\
Y(y) &  & A_{y}\sin\left(\frac{n_{y}\pi y}{b}\right)\qquad(n_{y}=1,2,3\ldots)\qquad E_{y}=\frac{n_{y}^{2}\pi^{2}\hbar^{2}}{2mb^{2}}\\
Z(z) &  & A_{z}\sin\left(\frac{n_{z}\pi z}{c}\right)\qquad(n_{z}=1,2,3\ldots)\qquad E_{z}=\frac{n_{z}^{2}\pi^{2}\hbar^{2}}{2mc^{2}}
\end{eqnarray*}
Therefore the 3D solution is given by,
\[
\psi=XYZ=A\sin\left(\frac{n_{x}\pi x}{a}\right)\sin\left(\frac{n_{y}\pi y}{b}\right)\sin\left(\frac{n_{z}\pi z}{c}\right)
\]
with
\[
E=E_{x}+E_{y}+E_{z}=\frac{\pi^{2}\hbar^{2}}{2m}\left(\frac{n_{x}^{2}}{a^{2}}+\frac{n_{y}^{2}}{b^{2}}+\frac{n_{z}^{2}}{c^{2}}\right)
\]



\subsection{Regular Cube}

Consider the symmetric box (cube) with $a=b=c$, then
\[
E=\frac{\pi^{2}\hbar^{2}}{2ma^{2}}\left(n_{x}^{2}+n_{y}^{2}+n_{z}^{2}\right)
\]


\begin{minipage}[t]{1\columnwidth}%
\noindent \begin{center}
\includegraphics{\string"Quantum Mechanics/particle in a cube\string".pdf}
\par\end{center}%
\end{minipage}


\section{Energy Degeneration}

Use the particle in a symmetrical cube as an example. Different combinations
of $n_{x},n_{y},n_{z}$ can give the same value for the energy eigenvalue
of the particle. This is called the degeneracy, when there is more
than one distinct eigenfunctions that share the same numerical eigenvalue.

\begin{minipage}[t]{1\columnwidth}%
\noindent \begin{center}
\begin{tabular}{|c|c|c|c|c|c|}
\hline 
Energy Level & $n_{x}$ & $n_{y}$ & $n_{z}$ & $E$ $\left(\text{in units of }\frac{\pi^{2}\hbar^{2}}{2ma^{2}}\right)$ & Degeneracy Value\tabularnewline
\hline 
\hline 
Ground State & 1 & 1 & 1 & 3 & 1\tabularnewline
\hline 
\multirow{3}{*}{1st Excited state} & 2 & 1 & 1 & 6 & \multirow{3}{*}{3}\tabularnewline
\cline{2-5} 
 & 1 & 2 & 1 & 6 & \tabularnewline
\cline{2-5} 
 & 1 & 1 & 2 & 6 & \tabularnewline
\hline 
\multirow{3}{*}{2nd Excited State} & 2 & 2 & 1 & 9 & \multirow{3}{*}{3}\tabularnewline
\cline{2-5} 
 & 2 & 1 & 2 & 9 & \tabularnewline
\cline{2-5} 
 & 1 & 2 & 2 & 9 & \tabularnewline
\hline 
\end{tabular}
\par\end{center}%
\end{minipage}

This effect of degeneracy emerges when there is any form of symmetry
in the potential function of the system. Similarly, whenever energy
degeneracy is seen, there must be symmetry present in the potential
function. Degeneracy is closely related to the modeling of physical
systems, such as the hydrogen atoms.


\section{Angular Momentum}

\marginpar{E+R pg 254-262}[-1cm]

This is another case of new phenomena that emerges when the higher
dimensions are included, it cannot exist in 1D or 2D, it is an intrinsically
3D property. Classically,
\[
\underline{L}=\underline{r}\times\underline{p}\qquad\begin{cases}
\underline{L}_{x}=yp_{z}-zp_{y}\\
\underline{L}_{y}=zp_{x}-xp_{z}\\
\underline{L}_{z}=xp_{y}-yp_{x}
\end{cases}
\]
So the quantum mechanical operator is a combination of the position
and linear momentum operators,
\begin{eqnarray*}
\hat{L}_{x} & = & -i\hbar\left(y\frac{\partial}{\partial z}-z\frac{\partial}{\partial y}\right)\\
\hat{L}_{y} & = & -i\hbar\left(z\frac{\partial}{\partial x}-x\frac{\partial}{\partial z}\right)\\
\hat{L}_{z} & = & -i\hbar\left(x\frac{\partial}{\partial y}-y\frac{\partial}{\partial x}\right)
\end{eqnarray*}
There also exists an operator for the $\left(\text{angular momentum}\right)^{2}$,
\[
\hat{L}^{2}=\hat{L}_{x}^{2}+\hat{L}_{y}^{2}+\hat{L}_{z}^{2}
\]
So there are 4 quantum mechanical operators for the angular momentum
of a particle in three dimensions. This starts from the premise that
the position and the momentum are both known well, so that they can
be combined to form the angular momentum. however this is expressly
forbidden in quantum mechanics, due to the uncertainty product that
relates them $\Delta x\Delta p\geq\frac{\hbar}{2}$. Look at the operators
and their commutation,
\begin{eqnarray*}
\left[\hat{L}_{x},\hat{L}_{y}\right] & = & \hat{L}_{x}\hat{L}_{y}-\hat{L}_{y}\hat{L}_{x}\\
 & = & -\hbar^{2}\left[\left(y\frac{\partial}{\partial z}-z\frac{\partial}{\partial y}\right)\left(z\frac{\partial}{\partial x}-x\frac{\partial}{\partial z}\right)-\left(z\frac{\partial}{\partial x}-x\frac{\partial}{\partial z}\right)\left(y\frac{\partial}{\partial z}-z\frac{\partial}{\partial y}\right)\right]\\
 & = & -\hbar^{2}\left(y\frac{\partial}{\partial x}-x\frac{\partial}{\partial y}\right)\\
 & = & -\hbar^{2}\hat{L}_{z}
\end{eqnarray*}
So
\begin{eqnarray*}
\left[\hat{L}_{x},\hat{L}_{y}\right] & = & -\hbar^{2}\hat{L}_{z}\\
\left[\hat{L}_{y},\hat{L}_{z}\right] & = & -\hbar^{2}\hat{L}_{x}\\
\left[\hat{L}_{z},\hat{L}_{x}\right] & = & -\hbar^{2}\hat{L}_{y}
\end{eqnarray*}
So in general, the angular momentum operators do not commute, so we
can know the precise value of only one component of $\underline{L}$.
We cannot know $\underline{L}=(L_{x},L_{y},L_{z})$ fully.

Analyze $(\text{angular momentum})^{2}$ operator.
\[
\left[\hat{L}^{2},\hat{L}_{z}\right]=\left[\hat{L}_{x}^{2},\hat{L}_{z}\right]+\left[\hat{L}_{y}^{2},\hat{L}_{z}\right]+\left[\hat{L}_{z}^{2},\hat{L}_{z}\right]
\]
Since any operator always commutes with itself,
\begin{eqnarray*}
\left[\hat{L}_{z}^{2},\hat{L}_{z}\right] & = & 0\\
\left[\hat{L}_{x}^{2},\hat{L}_{z}\right] & = & \left[\hat{L}_{x}\hat{L}_{x},\hat{L}_{z}\right]
\end{eqnarray*}
Use commutator identity mentioned previously,
\[
\left[\hat{A}\hat{B},\hat{C}\right]=\hat{A}\left[\hat{B},\hat{C}\right]+\left[\hat{A},\hat{C}\right]\hat{B}
\]
\begin{eqnarray*}
\Rightarrow\left[\hat{L}_{x}^{2},\hat{L}_{z}\right] & = & \hat{L_{x}}\left[\hat{L_{x}},\hat{L_{z}}\right]+\left[\hat{L_{x}},\hat{L_{z}}\right]\hat{L_{x}}\\
 & = & \hat{L_{x}}\left(-i\hbar\hat{L_{y}}\right)+\left(-i\hbar\hat{L_{y}}\right)\hat{L_{x}}\\
 & = & -i\hbar\left(\hat{L}_{x}\hat{L}_{y}-\hat{L}_{y}\hat{L}_{x}\right)
\end{eqnarray*}
Similarly
\begin{eqnarray*}
\left[\hat{L}_{y}^{2},\hat{L}_{z}\right] & = & i\hbar\left(\hat{L}_{y}\hat{L}_{x}-\hat{L}_{x}\hat{L}_{y}\right)\\
\Rightarrow\left[\hat{L}_{x}^{2},\hat{L}_{z}\right] & = & -\left[\hat{L}_{y}^{2},\hat{L}_{z}\right]
\end{eqnarray*}
So 
\[
\left[\hat{L}^{2},\hat{L}_{z}\right]=0+0=0
\]
Similarly 
\begin{eqnarray*}
\left[\hat{L}^{2},\hat{L}_{z}\right] & = & \left[\hat{L}^{2},\hat{L}_{y}\right]=0
\end{eqnarray*}
So the $\left(\text{angular momentum}\right)^{2}$ commutes with all
three components of the angular momentum separately. Therefore we
can know the precise value of the $\left(\text{angular momentum}\right)^{2}$
\noun{and }one component only of angular momentum simultaneously.
By convention, we chose to know the z component, along with $\hat{L}^{2}$.
In other words, we can know the magnitude of the angular momentum,
$|\underline{L}|=\sqrt{L^{2}}$, but not the direction of this magnitude.


\section{Quantum Mechanics in Spherical Polar Co-ordinates}

\marginpar{E+R pg 235-238}[-1cm]

\begin{eqnarray*}
x & = & r\sin(\theta)\cos(\phi)\\
y & = & r\sin(\theta)\sin(\phi)\\
z & = & r\cos(\theta)
\end{eqnarray*}
$\theta$ is called the polar angle and $\phi$ is the azimuthal angle.
Using these, the equations for the angular momentum can be rewritten,
\begin{eqnarray*}
\hat{L}_{x} & = & i\hbar\left[\sin(\phi)\frac{\partial}{\partial\theta}+\cot(\theta)\cos(\phi)\frac{\partial}{\partial\phi}\right]\\
\hat{L}_{y} & = & i\hbar\left[-\cos(\phi)\frac{\partial}{\partial\theta}+\cot(\theta)\sin(\phi)\frac{\partial}{\partial\phi}\right]\\
\hat{L}_{z} & = & -i\hbar\frac{\partial}{\partial\phi}
\end{eqnarray*}
The rotation about the z-axis is independent of $\theta$ so the z-component
is simple.
\[
\hat{L^{2}}=-\hbar^{2}\left[\frac{1}{\sin(\theta)}\frac{\partial}{\partial\theta}\left(\sin(\theta)\frac{\partial}{\partial\theta}\right)+\frac{1}{\sin^{2}(\theta)}\frac{\partial^{2}}{\partial\theta^{2}}\right]
\]



\subsection{Eigenfunctions of $\hat{L_{z}}=-i\hbar\frac{\partial}{\partial\phi}$}

These have the form,
\begin{eqnarray*}
\Phi_{m} & = & Ae^{im\phi}\\
\hat{L_{z}\Phi_{m}} & = & -i\hbar\frac{\partial}{\partial\phi}\left(Ae^{im\phi}\right)=m\hbar\Phi_{m}
\end{eqnarray*}
$\Phi_{m}$ is an eigenfunction $L_{z}$ with an eigenvalue $m\hbar$.
The angle $\phi$ has a periodicity of $2\pi$. This leads to quantization
and normalization conditions. The period $=2\pi$ but $\phi+2\pi=\phi$.
So
\begin{eqnarray*}
\Phi(\phi+2\pi) & = & \Phi(\phi)\\
Ae^{im\phi} & = & Ae^{im(\phi+2\pi)}\\
\Rightarrow e^{2\pi im} & = & 1\\
\cos(2m\pi)+i\sin(2m\pi) & = & 1\qquad\therefore m\in\mathbb{Z},-\infty<m<\infty
\end{eqnarray*}
This is the boundary condition. The normalization condition is derived
from,
\begin{eqnarray*}
\int_{0}^{2\pi}\Phi_{m}^{2}d\phi & = & 1\\
\int_{0}^{2\pi}A^{2}d\phi & = & 1\\
A & = & \frac{1}{\sqrt{2\pi}}
\end{eqnarray*}
So the eigenfunctions of $\hat{L_{z}}$ are 
\[
\Phi_{m}(\phi)=\frac{1}{\sqrt{2\pi}}e^{im\phi}
\]
where $m\in\mathbb{Z}$ with eigenfunctions $L_{z}=m\hbar$. As $m$
can only take integer values, the value of the eigenvalues $L_{z}$
are quantized in units of $\hbar$.


\subsection{Eigenvalues of $(\text{Angular Momentum})^{2}$}

Can show that $L^{2}$ eigenvalues have the form,
\[
L^{2}=l(l+1)\hbar^{2}
\]
where l is the quantum number, $l\in\mathbb{Z}\geq0$. The quantum
number $l$ and $m$ are related, so write $m$ as $m_{l}$. Logically,
then, 
\[
L_{z}^{2}\leq L^{2}
\]
So
\[
\left(m_{l}\hbar\right)^{2}\leq l(l+1)\hbar^{2}
\]
But both $l$ and $m$ are required to be integers, so the maximum
value of $|m_{l}|$is $|m_{l}|=l$ (because the next integer value
$m_{l}=l+1\rightarrow m_{l}^{2}=(l+1)^{2}>l(l+1)$ which doesn't satisfy
the condition so must be incorrect). So the condition becomes
\[
|m_{l}|\leq l
\]
ie $m_{l}=-l,(-l+1),\ldots,0,\ldots,(l-1),l$. Note that $|m_{l}|\leq l$
implies that
\begin{eqnarray*}
L_{z}^{2} & < & L^{2}\\
L_{z}^{2} & \neq & L^{2}\ (\text{unless }l=0)
\end{eqnarray*}
So the angular momentum cannot be aligned directly along the z-axis.
This is because of the uncertainty product in angular momentum, as
in this case, all of the information about the angular momentum would
be known as, $L_{x}=L_{y}=0$.


\subsection{Visualization}

\uline{\noun{Ex}}

Let $l=2$, so $L^{2}=2(2+1)\hbar^{2}=6\hbar^{2}$ and $|L|=\sqrt{6}\hbar$.
Also 
\begin{eqnarray*}
m_{l} & = & -2,-1,0,1,2\\
L_{z} & = & -2\hbar,-\hbar,0,\hbar,2\hbar
\end{eqnarray*}
The length of the vector can then be known, but the location around
the contour cannot be known.

\begin{minipage}[t]{1\columnwidth}%
\noindent \begin{center}
\includegraphics{\string"Quantum Mechanics/angular momentum quantization\string".pdf}
\par\end{center}%
\end{minipage}


\subsubsection{Reduced Mass}

\marginpar{E+R pg 253-254}[-1cm]

A hydrogen atom has electrons, with mass $m$, and a very heavy nucleus
(proton) with mass $M$ with a separation $r$ which are both orbiting
a common center of mass. Kinematically this is the same as an electron
of reduced mass $\mu$orbiting a fixed nucleus (mass $\infty$) at
the same distance $r$, where
\[
\mu=\frac{mM}{m+M}
\]
This is the same as the classical situation.


\subsection{Single Electron Atoms}

Coulomb potential causes an attraction between the electron and the
nucleus,
\[
V(\underline{r})=\frac{q_{1}q_{2}}{4\pi\epsilon_{0}r}\qquad F(\underline{r})=\frac{q_{1}q_{2}}{4\pi\epsilon_{0}r^{2}}
\]
where $q_{1}=Ze$ ($Z$ is the atomic number, 1 for hydrogen) and
$q_{2}=-e$. This coulomb potential is an example of a radial or central
potential as it depends only on the separation,
\[
V(\underline{r})=V(r)
\]
so it is independent of $\theta$ and $\phi$.

TISE for an electron on an atom, therefore, is 
\[
\left(-\frac{\hbar^{2}}{2m}\nabla^{2}-\frac{Ze^{2}}{4\pi\epsilon_{0}r}\right)\psi(r,\theta,\phi)=E\psi(r,\theta,\phi)
\]
For the specific hydrogen atom case, $Z$ would equal 1. The Laplacian
operator in spherical polar co-ordinates is given by
\[
\nabla^{2}=\frac{1}{r^{2}}\frac{\partial}{\partial r}\left(r^{2}\frac{\partial}{\partial r}\right)+\frac{1}{r^{2}\sin(\theta)}\frac{\partial}{\partial\theta}\left(\sin(\theta)\frac{\partial}{\partial\theta}\right)+\frac{1}{r^{2}\sin^{2}(\theta)}\frac{\partial^{2}}{\partial\phi^{2}}
\]
Though this equation for $\nabla^{2}$ is more complicated than for
Cartesian co-ordinates, this form allows the equation to be solved
by separation of variables. Something that is not possible when in
Cartesian. Using this method, the result is that eigenfunctions are
found with the form
\[
\psi(r,\theta,\phi)=R(r)Y(\theta,\phi)
\]
which is composed of a radial part and an angular part. Th solution
of TISE also gives values for the energy eigenvalues. These are a
set of distinct energy eigenvalues of the form
\begin{eqnarray*}
E_{n} & = & \frac{-\mu Z^{2}e^{4}}{\left(4\pi\epsilon_{0}\right)^{2}2\hbar^{2}n^{2}}\\
E_{n} & \propto & -\frac{1}{n^{2}}\qquad n\in\mathbb{Z}
\end{eqnarray*}
So the electron is bound to the nucleus with discrete quantized energy
levels $E_{n}\propto-\frac{1}{n^{2}}$. For $Z=1$ (hydrogen atom),
\[
E_{n}=\frac{-13.6\unit{eV}}{n^{2}}
\]
The value of $E_{n}$ is negative because the potential is defined
to be $V=0$ at $r=\infty$, so any value of $r$ less than $\infty$
will produce a value of potential that is less than $0$. There exists
an infinite number of bound states but the energy separation tends
to zero as $r\rightarrow\infty$. $E=13.6\unit{eV}$ is the ionization
energy for the electron in the ground state of the hydrogen atom.


\subsubsection{Comparison of 3 Infinite Potential Wells}

\begin{minipage}[t]{1\columnwidth}%
\noindent \begin{center}
\includegraphics{\string"Quantum Mechanics/3 different wells\string".pdf}
\par\end{center}%
\end{minipage}

\begin{minipage}[t]{1\columnwidth}%
\noindent \begin{center}
\begin{tabular}{ccccc}
1D & Square Well & $V\propto r^{0}$ & $E\propto n^{2}$ & $n=1,2,3,\ldots$\tabularnewline
1D & SHO & $V\propto r^{2}$ & $E\propto\left(n+\frac{1}{2}\right)$ & $n=0,1,2,\ldots$\tabularnewline
3D & Coulomb & $V\propto r^{-1}$ & $E\propto-\frac{1}{n^{2}}$ & $n=1,2,3,\ldots$\tabularnewline
\end{tabular}
\par\end{center}%
\end{minipage}


\part{Single Electron Wave-functions}

\marginpar{E+R pg 242-254}[-1cm]


\section{Atomic Quantum Numbers and Polar Co-ordinates}

The most important quantum number is the one used here, called $n$,
the principle quantum number (sometimes also called the radial quantum
number or the shell quantum number). $n$ alone determines the energy
level $E_{n}\propto-\frac{1}{n^{2}}$. All eigenstates of the electron
within the atom with the same $n$ quantum number vale have the same
value. This is because of the energy degeneracy, and arises due to
the very symmetrical system in which the electrons are confined, ie
a spherical potential, $V(r)$. The radius (or , $<r>$) of the atom
increases along with the energy as the $n$ value increases. The solution
of TISE also restricts the angular momentum quantum number such that
\[
l<n\qquad(l\neq n)
\]
so
\[
0\leq l<n
\]
So for the three atomic quantum numbers, $n,l,m$ are all integers
\begin{itemize}
\item $n>0$, determines the energy $E_{n}\propto-\frac{1}{n^{2}}$ and
the radius
\item $0\leq l<n$, determines the magnitude of the orbital angular momentum
$|L|=\sqrt{l\left(l+1\right)\hbar^{2}}$
\item $-l\leq m_{l}\leq l$, determines the component of the orbital angular
momentum about the z-axis $L_{z}=m_{R}\hbar$
\end{itemize}
Set $n,l,m_{l}$ as the labels for the electron eigenstates in the
atom.
\[
\psi(r,\theta,\phi)=R_{n,l}(r)Y_{l,m}(\theta,\phi)
\]
$R_{n,l}(r)$ is the radial component that depends on the principle
quantum number, $n$, and the angular momentum quantum number, $l$,
and is a function only of the radial distance.

$Y_{l,m}(\theta,\phi)$ is the angular component that depends on the
angular momentum quantum number and the $m$ number and is a function
of the angles $\theta$ and $\phi$.


\subsection{Radial Wave-function}

This depends on the quantum numbers $n$ and $l$.
\[
R_{n,l}(r)=e^{-\frac{Zr}{na_{0}}}\left(\frac{Zr}{a_{0}}\right)^{l}G_{n,l}\left(\frac{Zr}{a_{0}}\right)
\]
where 
\begin{itemize}
\item $a_{0}=\frac{4\pi\epsilon_{0}\hbar}{\mu e^{2}}\approx0.529\times10^{-10}\unit{m}\approx\frac{\text{\AA}}{2}$
this is referred to as the Bohr radius, and is the ``radius'' of
a ground state hydrogen atom.
\item $G_{n,l}\left(\frac{Zr}{a_{0}}\right)$ is a polynomial function of
the order $n-(l+1)$
\item $e^{-\frac{Zr}{na_{0}}}$ tends to zero as r increases to infinity
since $n$ is on the denominator of the exponential. This means as
the energy levels increase, ie $n$ increases, the average radius
increases as the exponential penetrates further into the potential
wall.
\end{itemize}
The first few unnormalized states are given by,
\begin{eqnarray*}
R_{1,0} & \propto & e^{-\frac{Zr}{a_{0}}}\\
R_{2,0} & \propto & \left(1-\frac{Zr}{2a_{0}}\right)e^{-\frac{Zr}{2a_{0}}}\\
R_{2,1} & \propto & \frac{Zr}{a_{0}}e^{-\frac{Zr}{2a_{0}}}
\end{eqnarray*}
Hence we can evaluate the radial probability density functions,
\[
P(r)dr=|R_{nl}|^{2}\times4\pi r^{2}dr
\]
The final term comes from the area of the sphere,
\[
\int_{0}^{2\pi}rd\phi\int_{0}^{\pi}r\sin\theta d\theta dr
\]
\begin{minipage}[t]{1\columnwidth}%
\noindent \begin{center}
\includegraphics[scale=0.5]{\string"Quantum Mechanics/Radial Probability Density\string".pdf}
\par\end{center}%
\end{minipage}
\begin{itemize}
\item The average radius increases for higher values if $n$. The ground
state ($n=1$) peaks at $r=a_{0}$ (Bohr radius) and then the excited
states with $l=n-1$ peak at $r=n^{2}a_{0}$.
\item $P(r=0)=0$ due to the $4\pi r^{2}$ factor, but only for $l=0$ does
$R_{nl}(r=o)\neq0$.
\item The exponential factor tends to zero as the radius increases to infinity
as the wave-function penetrates into the wall of the potential barrier.
\item The number of nodes in the probability above $r=0$ is $n-(l+1)$
\end{itemize}
These characteristics determine much of the chemistry of the atom
as the control the location of the electrons in this single electron
atom.


\subsection{Angular Wave-function}

\[
Y_{l,m}(\theta,\phi)=(\text{polynomial in }\sin\theta\text{ or }\cos\theta)\times e^{im\phi}
\]
These go by the name ``spherical harmonic'' functions and are eigenfunctions
of the $\hat{L^{2}}$ operator. The angular probability density depends
only on $\theta$ since $|e^{im\phi}|^{2}=1$.

\begin{minipage}[t]{1\columnwidth}%
\noindent \begin{center}
\includegraphics[angle=270,scale=0.5]{\string"Quantum Mechanics/Polar Diagrams\string".pdf}
\par\end{center}%
\end{minipage}

\uline{\noun{Ex}}
\begin{enumerate}
\item $l=0$, $m_{l}=0$ $\Rightarrow Y_{0,0}(\theta,\phi)=1$ ie a spherical
symmetry without dependence on the angle.
\item $l=1$, $m_{l}=0$ $\Rightarrow Y_{1,0}(\theta,\phi)=\cos(\theta)e^{i\theta\times0}=\cos(\theta)$


Probability peaks along the z-axis $\theta=0$, $\theta=\pi$ because
$L_{z}=0$. ($L_{z}$ is angular momentum about the z-axis)


\begin{minipage}[t]{1\columnwidth}%
\noindent \begin{center}
\includegraphics[scale=0.6]{\string"Quantum Mechanics/polar diagram 1,0\string".pdf}
\par\end{center}%
\end{minipage}

\item $l=1$, $m_{l}\pm1$ $\Rightarrow Y_{1,\pm1}(\theta,\phi)=\sin(\theta)e^{\pm i\theta}$


Probability peaks at $\theta=\frac{\pi}{2}$ since $|L_{z}|=\hbar$


\begin{minipage}[t]{1\columnwidth}%
\noindent \begin{center}
\includegraphics[scale=0.6]{\string"Quantum Mechanics/polar diagram 1,1\string".pdf}
\par\end{center}%
\end{minipage}

\end{enumerate}
The time independent Schrödinger equation predicts energy and angular
momentum properties of an atomic electron. The energies are measured
with spectroscopy techniques, the results of which agree closely with
the predictions made by quantum mechanical calculations.


\section{Magnetic Dipole Moment}

An arbitrary electric charge is equivalent to a current loop. This
generates a magnetic dipole moment, $\mu$.

\begin{minipage}[t]{1\columnwidth}%
\noindent \begin{center}
\includegraphics[scale=0.7]{\string"Quantum Mechanics/magnetic dipole moment\string".pdf}
\par\end{center}%
\end{minipage}

\begin{eqnarray*}
\mu & = & \text{current}\times\text{area}\\
 & = & \frac{-e}{\left(\frac{2\pi r}{v}\right)}\times\pi r^{2}\\
 & = & \frac{-ver}{2}
\end{eqnarray*}
Also, the angular momentum, 
\begin{eqnarray*}
\underline{L} & = & \underline{r}\times\underline{p}\\
|\underline{L}| & = & rmv\\
\Rightarrow\mu & = & -\frac{eL}{2m}
\end{eqnarray*}
This is for, for example, an electron orbiting around a nucleus. Rewrite
this as,
\[
\underline{\mu_{l}}=-\frac{g_{l}\mu_{B}}{\hbar}\underline{L}
\]
where $l$ denotes the orbital angular momentum, $\mu_{B}=\frac{e\hbar}{2m}=0.927\times10^{-23}\unit[A]{m^{2}}$(this
is the Bohr magneton and is the typical unit used for the magnetic
dipole moment of an atom), $g_{l}=1$ (this is the orbital $g$ factor
for consistency with equation later on).

This was a purely classical electromagnetic argument, but the quantum
mechanical analysis gives the same equation. In quantum mechanics,
we can know the precise values of the magnitude and the $z$ component
of the dipole moment.\\
\begin{eqnarray*}
\text{Magnitude}\qquad|\mu_{l}| & = & \frac{g_{l}\mu_{B}}{\hbar}\sqrt{l(l+1)}\hbar\\
 & = & g_{l}\mu_{B}\sqrt{l(l+1)}\\
z\text{ component}\qquad(\mu_{l})_{z} & = & -\frac{g_{l}\mu_{B}}{\hbar}m_{l}\hbar\\
 & = & -g_{l}\mu_{B}m_{l}
\end{eqnarray*}
To measure the magnetic dipole, a non-uniform magnetic field is needed.

\begin{minipage}[t]{1\columnwidth}%
\noindent \begin{center}
\includegraphics{\string"Quantum Mechanics/non-uniform magnetic field\string".pdf}
\par\end{center}%
\end{minipage}

If the field lines were uniform, the force felt by the north and south
poles would be equal and opposite and so a torque would exist about
the center of the bar magnet, but there would exist no linear translation.
Since the field is non-uniform, the north pole feels a greater force
than the south pole.
\begin{eqnarray*}
\text{Potential energy}\qquad V & = & -\mu B\\
 & = & g_{l}\mu_{B}m_{l}B
\end{eqnarray*}
If $B_{z}$ varies along the $z$ axis, then
\[
F_{z}=-\frac{\partial V}{\partial z}=-g_{l}\mu_{B}m_{l}\frac{\partial B_{z}}{\partial z}
\]
This then gives a means to measure $m_{l}$ and hence get $L_{z}$
of the electron.


\subsection{Stern-Gerlach Experiment (1922)}

\marginpar{E+R pg 267-278}[-1cm]

\begin{minipage}[t]{1\columnwidth}%
\noindent \begin{center}
\includegraphics[scale=0.9]{\string"Quantum Mechanics/Stern Gerlach\string".pdf}
\par\end{center}%
\end{minipage}

\begin{minipage}[t]{1\columnwidth}%
\noindent \begin{center}
\includegraphics[angle=270,scale=0.5]{\string"Quantum Mechanics/Stern Geralch setup\string".pdf}
\par\end{center}%
\end{minipage}


\subsubsection*{Predicted Results}

For classical (unquantified) system, the angular momentum would be
expected to be continuous with a range of $L_{z}$ values,
\[
-L\leq L_{z}\leq L
\]
This would give a range of all possible results when measured, so
a stripe would be seen on the screen. For the quantized quantum mechanical
system, there are only certain discrete values of $L_{z}$, e.g. if
$L=1$, then $L_{z}=-\hbar,\,0,\,\hbar$ so there would exist only
three different spots on the screen from these possible values of
the angular momentum. Generally there are $2l+1$ values of $m_{l}$
so the expected number of spots to be seen would be $2l+1$. Since
$l$ is an integer, and odd number of spots should be seen.


\subsubsection*{Measured Results}

Consider the simplest case of a hydrogen atom in it's ground state
energy level, so $n=1$, $l=0$, $m_{l}=0$, $L_{z}=0$. Expected
results would be $\mu_{z}=0$ so no deflections and only one spot.
But this is not what was measured. The result observed was \emph{two}
spots with an equal deflection in the positive and negative directions
corresponding to $\mu_{z}=\pm\mu_{B}$.


\subsection{Conclusion}

Angular momentum \emph{is} quantized so no stripe is observed, but
Schrödinger equation does not correctly predict $\mu_{z}$.


\subsection{Spin}

The explanation of this observed effect is that the electron has an
intrinsic angular momentum, called spin. This has similar quantum
mechanical properties to the orbital angular momentum. This is impossible
to visualize as the electron is believed to have no physically dimensional
extent so cannot have a property related to rotation, However it is
a property that all electrons possess.

In quantum mechanics we can know,
\begin{eqnarray*}
\text{Magnitude}\qquad|\underline{S}| & = & \sqrt{S(S+1)}\hbar\\
z\text{-component}\qquad S_{z} & = & m_{s}\hbar
\end{eqnarray*}
where $|m_{s}|\leq S$ and is quantized in steps of 1. This leads
to the spin magnetic dipole moment,
\[
\left(\mu_{s}\right)_{z}=-g_{s}\mu m_{s}
\]
This leads to the conclusion that, because there are only two spots
and so there must be two possible values of the spin with an integer
step between them, that $m_{s}=\pm\frac{1}{2}$ and so $S_{z}=\pm\frac{1}{2}\hbar$.
This leads to the zero possibility of an electron ever having zero
spin. These two states are often called spin up and spin down, and
are the only spin states for a half spin particle.

Also, to get the observed deflection, $g_{s}=2=2g_{l}$. This spin
property is not predicted by the Schrödinger equation but is compatible
with it. However it is predicted from relativistic quantum mechanics
via the Dirac equation. This form \emph{requires }two components of
the electron wave function relating to the spin up and spin down states.
It also \emph{requires }that $g_{s}=2$ (as predicted from the Stern-Gerlach
experiment). Finally, it also includes two further components of the
wave-function that relate to the equivalent two spin possibilities
of the positron (anti-electron) which has the same $\frac{1}{2}$
spin characteristics as the electron.


\section{Total Angular Momentum}

\marginpar{E+R pg 281-284}[-1cm]

An atomic electron has both intrinsic (spin) and orbital angular momentum.
These can be combined into the total angular momentum,
\[
\underline{J}=\underline{S}+\underline{L}
\]
Associated with this property, there are similar total angular momentum
operators, $\hat{J_{x}}$, $\hat{J_{y}}$, $\hat{J_{z}}$, $\hat{J^{2}}$
and quantum numbers $j$ and $m_{j}$.
\begin{eqnarray*}
\text{Magnitude}\qquad|\underline{J}| & = & \sqrt{j(j+1)}\hbar\\
z\text{-component}\qquad J_{z} & = & m_{j}\hbar
\end{eqnarray*}
where $-j\leq m_{j}\leq j$ with quantization into unit steps. The
$z$-components are scalars, so $J_{z}=L_{z}+S_{z}$ and so $m_{j}=m_{l}+m_{s}$. 

For the $s=\frac{1}{2}$ case, the vector addition is simple as there
are only two possible values of $j$,
\begin{eqnarray*}
\text{Either}\qquad j & = & l+s\qquad(\underline{L},\underline{S}\text{ are parallel})\\
\text{Or}\qquad j & = & l-s\qquad(\underline{L},\underline{S}\text{ are anti-parallel})
\end{eqnarray*}
So
\[
j=l\pm\frac{1}{2}
\]
(if $l=0$ then only $j=\frac{1}{2}$ is possible as $j>0$)

\uline{\noun{Ex}}

Electron in one of the $l=1$ atomic states.
\[
\begin{array}{cccc}
\text{So when }s=+\frac{1}{2} & {\displaystyle j=1+\frac{1}{2}=\frac{3}{2}} & \text{Or when }s=-\frac{1}{2} & {\displaystyle j=1-\frac{1}{2}=\frac{1}{2}}\\
 & {\displaystyle |\underline{J}|=\sqrt{\frac{3}{2}\times\frac{5}{2}}\hbar=\frac{\sqrt{15}}{2}\hbar} &  & {\displaystyle |\underline{J}|=\sqrt{\frac{1}{2}\times\frac{3}{2}}\hbar=\frac{\sqrt{3}}{2}\hbar}\\
 & {\displaystyle m_{j}=-\frac{3}{2},-\frac{1}{2},\frac{1}{2},\frac{3}{2}} &  & {\displaystyle m_{j}=-\frac{1}{2},\frac{1}{2}}
\end{array}
\]


In general, the multiplicity of $m_{j}=2j+1$. Since $m_{j}=m_{l}+m_{s}$,
if a value of $m_{l}$ is already know, some of the values of $m_{j}$
might be ruled out.


\section{Pauli Exclusion Principle}

There are two distinct types of quantum particle,
\begin{itemize}
\item Bosons
\item Fermions
\end{itemize}

\subsubsection*{Bosons}

Bosons are particles with integer spin, $s=0,1,2,3\ldots$, $s\in\mathbb{Z}$,
e.g. photon, pion, $^{4}\text{He}$ nucleus, $\text{W}^{\pm}$, $\text{Z}^{0}$.
Importantly, for bosons, an unlimited number of them can occupy the
same quantum state simultaneously. This leads to important phenomena
like lasers, super-conductivity, super-fluidity, Bose-Einstein condensates
etc.


\subsubsection*{Fermions}

Fermions are half integer spin particles, $s=\frac{1}{2},\frac{3}{2},\frac{5}{2}\ldots$,
$s\in\frac{\mathbb{Z}}{2}$, e.g. electron, proton, quark, $^{3}\text{He}$
nucleus etc. The Pauli exclusion principle states that no two fermions
can occupy the same quantum state simultaneously. Alternatively, no
two fermions can share exactly the same set of quantum numbers. This
is essential for multi-electron atoms as it means the electrons cannot
all occupy the ground state, and so leads to the field of chemistry.


\section{Atomic Quantum Numbers}

We now need 4 quantum numbers to specify the eigenstate of the electron,
\[
n,l,m_{l},m_{s}
\]
$m_{s}=\pm\frac{1}{2}$, gives the $z$-component of the spin angular
momentum.

Alternative set of 4 quantum numbers,
\[
n,l,j,m_{j}
\]
$j,m_{j}$ give the magnitude and $z$-component of the total angular
momentum. The first set is fine for single electron atoms, but this
leads to ambiguity when multi-electrons are considered. So the second
set is used to reduce ambiguity and simplify definitions.


\part{Multi-Electron Atoms}

The Pauli Exclusion Principle requires that, when more than one electron
is present, they occupy different eigenstates. For the ground state
atom, the lowest states are occupied first, that is, the lowest $n$-value
and lowest $l$-value are filled first. As more eigenstates are filled,
the different elements with different chemical properties are formed.

\noindent \begin{center}
\begin{tabular}{lllll}
$n=1$ & $l=0$ & $m_{l}=0$ & $m_{s}=\pm\frac{1}{2}$ & 2 states ($\text{H},\text{He}$)\tabularnewline
$n=2$ & $l=0$ & $m_{l}=0$ & $m_{s}=\pm\frac{1}{2}$ & \multirow{2}{*}{8 states ($\text{Li}\rightarrow\text{Na}$)}\tabularnewline
 & $l=1$ & $m_{l}=0,\pm1$ & $m_{s}=\pm\frac{1}{2}$ & \tabularnewline
$n=3$ & $l=0$ & $m_{l}=0$ & $m_{s}=\pm\frac{1}{2}$ & \multirow{3}{*}{18 states ($\text{Na}\rightarrow$)}\tabularnewline
\multirow{2}{*}{} & $l=1$ & $m_{l}=0,\pm1$ & $m_{s}=\pm\frac{1}{2}$ & \tabularnewline
 & $l=2$ & $m_{l}=0,\pm1,\pm2$ & $m_{s}=\pm\frac{1}{2}$ & \tabularnewline
\end{tabular}
\par\end{center}

The energy levels are disrupted for $n=3$ due to the effect of the
other electrons in the atom and so the higher level energy states
do not follow this simple hydrogen atom model.

The even spots observed in the Stern-Gerlach experiment is due to
the contribution when the is an odd number of electrons since this
case is when there is an unpaired electron.


\subsection{Spectroscopic Notation}

It is conventional to write the atomic quantum numbers in the form
\[
n[l]_{j}
\]
where $n$ and $j$ are numerals, but $[l]$ is a letter code to signify
the value of $l$,
\begin{eqnarray*}
l=0 &  & s\qquad\text{(strong)}\\
l=1 &  & p\qquad\text{(primary)}\\
l=2 &  & d\qquad\text{(diffuse)}\\
l=3 &  & f\qquad\text{(fine)}
\end{eqnarray*}
e.g. 
\begin{eqnarray*}
1s_{\frac{1}{2}}, &  & n=1,\; l=0,\; j=\frac{1}{2},\;\left(m_{j}=\pm\frac{1}{2}\right)\\
2s_{\frac{3}{2}}, &  & n=2,\; l=1,\; j=\frac{3}{2},\;\left(m_{j}=\pm\frac{1}{2},\pm\frac{3}{2}\right)
\end{eqnarray*}



\subsection{Selection Rules for Atomic Transitions}

\marginpar{E+R pg 288-295}[-1cm]

Atomic electrons can transition between one state and another, emitting
or absorbing a photon of energy equal to the difference between the
energy levels. This is the field of spectroscopy. Only certain transitions
are allowed, obeying certain selection rules:
\begin{itemize}
\item $\Delta l$ must be $\pm1$ \noun{and}
\item $\Delta j$ must be $0$ or $\pm1$
\end{itemize}
These rules arise from the photon properties and conservation laws,
\begin{itemize}
\item $\Delta j=0,\pm1$ because a photon is a spin $s=1$ particle, so
carries a unit angular momentum, so to conserve angular momentum,
the atom must change by one unit. So either $\Delta j=\pm1$ or $j=0$
and $m_{j}$ changes by one unit, eg $m_{j}=-\frac{1}{2}\leftrightarrow m_{j}=+\frac{1}{2}$,
this is when the spin flips.
\item $\Delta l=\pm1$ comes from the property of parity. The photon has
a negative intrinsic parity which must be conserved. So when the photon
is emitted or absorbed, the parity of the atom must change. $\text{Atomic parity}=(-1)^{l}$
(from symmetry of angular wave-function, $Y_{l,m}$) So $\Delta l=\pm1$
changes parity.
\end{itemize}
\uline{\noun{Ex}}

\begin{eqnarray*}
1s_{\frac{1}{2}}\rightarrow2p_{\frac{1}{2}} &  & \qquad\Delta l=1,\;\Delta j=0\qquad\checkmark\\
1s_{\frac{1}{2}}\rightarrow2p_{\frac{3}{2}} &  & \qquad\Delta l=1,\;\Delta j=1\qquad\checkmark\\
2s_{\frac{1}{2}}\rightarrow1s_{\frac{1}{2}} &  & \qquad\Delta l=0,\;\Delta j=0\qquad\times\\
3d_{\frac{5}{2}}\rightarrow2p_{\frac{3}{2}} &  & \qquad\Delta l=-1,\;\Delta j=-1\qquad\checkmark
\end{eqnarray*}


\pagebreak{}


\part*{Appendix}

\addcontentsline{toc}{part}{Appendix}

\appendix


\section{Operators}

Momentum
\begin{eqnarray*}
\hat{p_{x}} & = & -i\hbar\frac{\partial}{\partial x}\\
\hat{p} & = & -i\hbar\underline{\nabla}
\end{eqnarray*}


Hamiltonian
\begin{eqnarray*}
\hat{\mathcal{H}_{x}} & = & -\frac{\hbar^{2}}{2m}\frac{\partial^{2}}{\partial x^{2}}+V(x)\\
\hat{\mathcal{H}} & = & -\frac{\hbar^{2}}{2m}\nabla^{2}+V(x,y,z)
\end{eqnarray*}


Energy
\[
\hat{E}=i\hbar\frac{\partial}{\partial t}
\]


Angular Momentum

\begin{eqnarray*}
\hat{L}_{x} & = & -i\hbar\left(y\frac{\partial}{\partial z}-z\frac{\partial}{\partial y}\right)\\
\hat{L}_{y} & = & -i\hbar\left(z\frac{\partial}{\partial x}-x\frac{\partial}{\partial z}\right)\\
\hat{L}_{z} & = & -i\hbar\left(x\frac{\partial}{\partial y}-y\frac{\partial}{\partial x}\right)
\end{eqnarray*}


\[
\hat{L}^{2}=\hat{L}_{x}^{2}+\hat{L}_{y}^{2}+\hat{L}_{z}^{2}
\]


\addcontentsline{toc}{section}{B Ladder Operator Solution of the Harmonic Oscillator}

\includepdf{\string"Quantum Mechanics/\string"ladder-operators}

\addcontentsline{toc}{section}{C Calculating the Transmission Coefficient for a Potential Basrrier}

\includepdf[pages=-]{\string"Quantum Mechanics/\string"barrier-transmission-handout}

\addcontentsline{toc}{section}{D Solution of TISE for a Finite 1D Square-Well Potential}

\includepdf[pages=-]{\string"Quantum Mechanics/\string"finite-well}

\addcontentsline{toc}{section}{E Properties of Commutators}

\includepdf[pages=-]{\string"Quantum Mechanics/\string"commutator-algebra}

\addcontentsline{toc}{section}{F Significance of Commuting Operators}

\includepdf[pages=-]{\string"Quantum Mechanics/\string"commutation-and-eigenfunctions}
\end{document}
